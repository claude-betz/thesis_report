\chapter{Literature Review}

\section{Definition of Important terms}
The field of Computer vision is riddled with similar sounding terms whose intent
can get muddled during the discussion of concepts.  The purpose of this section
is to establish the intent of these keywords as they are used later in this text.

Detailed below are definitions of important terms within the field of computer
vision, some of which are used rather loosely within the literature or are
despite seeming interchangeable may refer to different concepts. 

Feature Detection:

Image Segmentation: This is the process of partitioning an image into multiple
sets of pixels with the goal of making the image easier to analyse, it typically
involves finding boundary lines of objects. 

Motion Segmentation: Motion Segmentation refers to the labelling of pixels that
are associated with each independently moving 3D object in sequence of images
that can feature multiple motions. \cite{Tekalp2014}

Object Recognition: This a method by which Given an image, the objects within
said image are detected and classified as one of a set of predefined Object
Category.  I.e. “What Object(s) are in this frame?”

Object Detection: Object Detection can be seen as targeted recognition, it deals
with finding instances of objects within a given image or video frame, this
technique usually makes use of extracted features and learning algorithms to
recognize instances of an object category.  I.e. “Where is this particular
object in this frame?”

Object Classification: This is a broader treatments than Detection, where a
system is not only able to identify objects within images but can differentiate
between and label different classes of objects within the image. 

Optical Flow: refers to the pattern of apparent motion of objects, surfaces, and
edges in a visual scene caused by the relative motion between an observer and a
scene. Optic flow fields can be used to structure or segment a scene. [3]

ptical Flow Segmentation: Refers to the grouping together of optical flow
vectors that are associated with the same 3-D motion or structure. This problem
is identical to motion segmentation provided a dense optical flow field. [1]

Object Tracking: Object tracking also known as Video Tracking refers to the use
of sensor measurements to determine the location, path and characteristics of
objects of interest. [4]
 
Motion Estimation: A fundamental problem in video processing that is concerned
with determining motion vectors that describe the transformation of one 2D
image to another over time. The problem can be broken down into
\cite{Tekalp2014};

2D Motion Estimation: Defined as an “ill posed problem”, as motion is in
three dimensions but image worked on are projections of 3D phonomena onto a
2D image plane.

NOTE:
Motion Tracking: In the literature, this may also refer to Motion Capture;
The technology of recording movements of one objects and applying said
tracking data to another object.\ e.g.\ animation. Or as it is used in the
case of this paper Video/Object Tracking.


