\chapter{Literature Review}\label{literature_review}

This Literature Review begins by establishing key definitions necessary for navigating
the literature. This is followed by a brief recap of how the field of Computer
Vision and more specifically Motion Tracking came to be.

The Review then broadly details the general approach to solving of Motion Tracking
problems. This is followed by a study of the progression of the field,
specifically reviewing prominent “classical” pre-deep learning approaches to the
problem.

\section{Establishing Key Terms}
The field of Computer vision is riddled with similar sounding terms whose intent
can get muddled during the discussion of concepts.  The purpose of this section
is to establish the intent of these keywords as they are used later in this text.

Detailed below are definitions of important terms within the field of computer
vision, some of which are used rather loosely within the literature or are
despite seeming interchangeable may refer to different concepts. 

Feature Detection:

Image Segmentation: This is the process of partitioning an image into multiple
sets of pixels with the goal of making the image easier to analyse, it typically
involves finding boundary lines of objects. 

Motion Segmentation: Motion Segmentation refers to the labelling of pixels that
are associated with each independently moving 3D object in sequence of images
that can feature multiple motions.~\cite{Tekalp2014}

Object Recognition: This a method by which Given an image, the objects within
said image are detected and classified as one of a set of predefined Object
Category.  I.e. “What Object(s) are in this frame?”

Object Detection: Object Detection can be seen as targeted recognition, it deals
with finding instances of objects within a given image or video frame, this
technique usually makes use of extracted features and learning algorithms to
recognize instances of an object category.  I.e. “Where is this particular
object in this frame?”

Object Classification: This is a broader treatments than Detection, where a
system is not only able to identify objects within images but can differentiate
between and label different classes of objects within the image. 

Optical Flow: refers to the pattern of apparent motion of objects, surfaces and
edges in a visual scene caused by the relative motion between an observer and a
scene. Optic flow fields can be used to structure or segment a scene~\cite{Forsyth2012}.

Optical Flow Segmentation: Refers to the grouping together of optical flow
vectors that are associated with the same 3-D motion or structure. This problem
is identical to motion segmentation provided a dense optical flow field~\cite{Tekalp2014}.

Object Tracking: Object tracking also known as Video Tracking refers to the use
of sensor measurements to determine the location, path and characteristics of
objects of interest~\cite{Challa2011}.
 
Motion Estimation: A fundamental problem in video processing that is concerned
with determining motion vectors that describe the transformation of one 2D
image to another over time~\cite{Tekalp2014}.

NOTE:\@
tracking data to another object.\ e.g.\ animation. Or as it is used in the
case of this paper Video/Object Tracking.

\section{Brief History of Computer Vision and Motion Tracking}
*insert anecdote*


\section{Motion Tracking}
The problem of isolating moving objects of interest from a scene can
be categorised under the branch of Computer Vision known as Motion
Estimation/Tracking.

\subsection{The Representation of Video}
While a still image has a spatial distribution of intensity that is constant
with time, videos can be defined as a time-varying image whose spatial
distribution of intensity changes with respect to time, we can denote this by
$s_c(x_1,x_2,t)$ where $x_1$ and $x_2$ represent spatial variables and $t$ is
the temporal variable. \cite{Tekalp2014}

Video can be represented either in analog form.
Typically video recording, storage and transmission used to be in analog form
i.e.\ recording on video cassettes, reception by antenna,  

\subsection{A General Approach to Motion Tracking}\label{literature_review_general_approach}
The problem of Motion Tracking can coarsely be broken down into a sequence of
subtasks, these roughly being:
    \begin{itemize}
        \item Object Detection: This is concerned with (initially) identifying
            whether a, or multiple object is/are present within a given scenario.
        \item Object Classification: 
        \item Object Tracking: This is concerned with successively following the
            identified objects in subsequent frames in the provided video.
    \end{itemize}
The relevant literature details numerous approaches to each of the above
subtasks of the overall Motion Tracking Task. 
These approaches to the three subtasks specified above can conveniently be
presented as a taxonomy as shown in the  

*insert figure of taxonomy of motion tracking sub task approaches*
 
\subsection{Object Detection}

\subsection{Object Classification}

\subsection{Object Tracking}


\section{Software}

\subsection{Relevant Languages}
\subsection{Relevant Libraries}




