\chapter{Theoretical Framework}

This Chapter aims to provide a comprehensive treatment of the theory necessary
for the implementation of the Overall MOT System. This is in line with the
objectives of the study outlined in Section \ref{introduction_objectives} and
the research direction determined in reviewing the relevant literature in
Chapter \ref{literature_review}.

\section{Mean Shift Tracker}
As summarised in section \ref{literature_review_mean_shift}, the Mean Shift Tracker was
first proposed by Commaniciu et al. It is addresses the
visual tracking problem of Target Representation and Localization.

The idea of this tracker is that spatially masking an object with an
isotropic kernel allows for the definition of a spatially-smooth similarity
function between a target and potential candidate model. This smooth similarity
allows for gradient-based optimization in achieving target localisation in
subsequent frames \cite{Comaniciu2003}.

A description of theory underlying the Mean Shift Tracker follows:

\subsection{Model Representation}
To represent a particular target or candidate Model, a suitable feature space
must be chosen. The choices could be anything from grayscale intensity, RGB
intensity or Texture of the particular object. Objects geometrically described 
as elliptical sub-domains $\mathbf{D}$ of a frame, with dimensions $h_x$ and
$h_y$ and center $\mathbf{x}$.

Target and candidate models are defined as pdf's within the chosen feature
space defined over $\mathbf{D}$. In the initial frame, $\mathbf{f}_k$, a target
model of the object to be tracked located at $\mathbf{x_0}$ is defined by it's
pdf, $\hat{q}(\mathbf{x_0})$. In the subsequent frame, $\mathbf{f_{k+1}}$, a
candidate model is defined with pdf $\hat{p}(\mathbf{x_0})$ starting at the last
known object position $\mathbf{x_0}$. 

Both pdf's $\hat{q}(\mathbf{x_0})$ and $\hat{p}(\mathbf{x_0})$ are estimated from
their respective frames $\mathbf{f}_k$ and $\mathbf{f}_{k+1}$ using $m$-bin
histograms in the relevant feature space. Depending on the number of
features used computational complexity of calculating the histogram and space
needed for storing the histogram goes up by a factor of $m$, as we are
increasing the dimensionality of our feature space by 1 for each additional
feature thus incrementing the size by a factor of $m$-bins.

The models for the target and candidate are summarised below:
$$\text{target model: } \hat{q}(\mathbf{x_0})=\{\hat{q}_u(\mathbf{x_0})\}_{u=1,...,m}$$
$$\text{candidate model: } \hat{p}(\mathbf{x_i})=\{\hat{p}_u(\mathbf{x_i})\}_{u=1,...,m}$$

All model/candidate histograms are normalized. Therefore, a general m-bin
histogram, $\hat{p}(\mathbf{x_i})$ defined over spacial sub-domain, $\mathbf{D}$
within a frame $\mathbf{f}_k$, satisfies the following normalisation condition:
\begin{equation}\label{eqn_histogram_normalisation}
    \sum_{u=1}^{m}\hat{p}_u = 1
\end{equation}

Computing the histogram occurs according to the following equation:
\begin{equation}
    \text{pdf}(u)=C\sum_{i=1}^{n_x}\kappa({\parallel{\mathbf{x_i}^*}-{\mathbf{x_0}^*}\parallel}^2)\delta[b(\mathbf{x_i}^*)-u]    
\end{equation}
\begin{equation}
    C=\frac{1}{\sum_{i=1}^{n_x}\kappa({\parallel\mathbf{x_i}^*\parallel}^2)}
\end{equation}
where:
\begin{itemize}
    \item $\parallel{\mathbf{x_i}}^*-{\mathbf{x_0}}^*\parallel$: The euclidean
        norm between $\mathbf{x_i}^*$ and ${\mathbf{x_0}}^*$ squared which
        reduces to $\parallel{\mathbf{x_i}}^*\parallel$ for $\mathbf{x_0} = 0$. 
    \item $\kappa({\parallel{\mathbf{x_i}}^*\parallel}^2)$: Kernel function that
        defining weights based on $\parallel{\mathbf{x_i}}^*\parallel^2$.
    \item $b({\mathbf{x_i}}^*)$: A function which maps the normalized pixel
        values in $\mathbf{n_x}$ to their RGB pixel value in $\mathbf{D}$.
    \item $\delta[b({\mathbf{x_i}}^*)-u]$: Abstraction to say only evaluate for bin u.
    \item $C$: Normalisation constant to ensure histogram satisfies equation
        \ref{eqn_histogram_normalisation}.
\end{itemize}

\subsection{Kernel Profile, $\kappa(x)$: Epanechnikov Kernel}
A Kernel is a weighting function employed in non-parametric density estimation. 

The recommended kernel function for the mean shift tracker is that of the
Epanechnikov Kernel.

The Epanechnikov Kernel has the following general equation:
\begin{equation}
    \kappa(x)=\frac{1}{2 c_d}(d+2)(1-x), |x|<1
\end{equation}
where:
\begin{itemize}
   \item $d$: the dimensions of the kernel
   \item $c_d$: the area of a unit circle in the dimension $d$
   \item $x$: the square of the Euclidian distance of the normalized pixel from $\mathbf{D}$
\end{itemize}

Of interest is the 2D variant of this equation given that the spatial smoothing
is occurring in the 2D $(x,y)$ pixel plane. Thus we choose $d=2$ and $c_d=\pi$.
As the kernel profile does not vary for a particular region of interest, it is
computationally convenient to pre-calculate its values for use in the subsequent
derivation of the subsequent histogram

The calculation of the Epanechnikov Kernel Matrix for a particular $\mathbf{D}$ is
illustrated in Figure \ref{epanechnikov_kernel}.

\Figure[width=0.8\columnwidth]{Illustration of Epanechnikov Matrix Derivation\label{epanechnikov_kernel}}{design_epanechnikov_kernel_illustration}

The pixels in the elliptical mask of the target/candidate are normalized to a unit
circle by scaling the ellipse by $h_y$ and $h_x$ (which are the major and minor
axes size of the ellipse). Each pixel is then assigned a weight according to the
value of the 2D Epanechnikov Kernel for said pixel's normalized euclidean
distance from the unit circle centre as illustrated in Figure \ref{epanechnikov_kernel}.

\subsection{Similarity Measure, $\rho$: Bhattacharyya Coefficient}

$\rho_i \equiv \rho[\hat{q}(\mathbf{x_0}),\hat{p}(\mathbf{x_i})]$
Given a target model's pdf $\hat{q}(\mathbf{x_0})$, and a candidate pdf,
$\hat{p}(\mathbf{x_i})$, in a subsequent frame.  The function $\rho_i$ returns a
metric describing how strongly a target $\hat{p}(\mathbf{x_i})$ resembles
$\hat{q}(\mathbf{x_0})$.

As per \cite{Comaniciu2003} we make use of the Bhattacharyya coefficient, $BC$ as our
measure of similarity measure $\rho$ between the target and candidate pdfs. The
Bhattacharyya coefficient is defined by the following equation:
\begin{equation}
    BC = \rho[\hat{p}(y),\hat{q}]=\sum_{u=1}^{m}\sqrt{\hat{p}_u(y)\hat{q}_u}
\end{equation}

\subsection{The General Algorithm}
The general algorithm for Mean Shift Tracking between two frames is defined by
flow chart in Figure \ref{mean_shift_tracking_algorithm}. 

\Figure[width=0.7\columnwidth]{Flow Chart of Mean Shift Tracking Algorithm
between two frames $\mathbf{f}_k$ and
$\mathbf{f}_{k+1}$\label{mean_shift_tracking_algorithm}}{design_mean_shift_tracking_algorithm}




