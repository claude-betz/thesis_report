\chapter{Results} \label{chapter_results}

This Chapter is concerned with the presentation of results of the application of
the algorithms at a modular level and the performance of the overall integrated system.

\section{Template Matching Tracker}
This Section presents with a qualitative analysis of both the adaptive and
non-adaptive Template Matching Trackers proposed in
Section~\ref{theoretical_framework_template_matching_trackers}. 

Both trackers were applied to the famous Hamburg taxi sequence \cite{} as an initial
test of their applicability to the Motion Tracking Problem.

The Template Trackers below were run for a threshold of 0.8.

\subsection{Simple Template Matching}
The initial template used is the shown by the rectangular region around the car
in the first frame. The images shown correspond to almost equally spaced
samples of the 40 frame taxi sequence.

\begin{figure} \label{fig:simple_template_tracking}
    \makebox[\linewidth][c]{
    \begin{tabular}{cccc}
        \subfloat{\includegraphics[width = 1.5in]{figures/results/simple_template_tracker/taxi_sequence/1.jpg}} &
        \subfloat{\includegraphics[width = 1.5in]{figures/results/simple_template_tracker/taxi_sequence/2.jpg}} &
        \subfloat{\includegraphics[width = 1.5in]{figures/results/simple_template_tracker/taxi_sequence/3.jpg}} &
        \subfloat{\includegraphics[width = 1.5in]{figures/results/simple_template_tracker/taxi_sequence/4.jpg}} \\

        \subfloat{\includegraphics[width = 1.5in]{figures/results/simple_template_tracker/taxi_sequence/5.jpg}} &
        \subfloat{\includegraphics[width = 1.5in]{figures/results/simple_template_tracker/taxi_sequence/6.jpg}} &
        \subfloat{\includegraphics[width = 1.5in]{figures/results/simple_template_tracker/taxi_sequence/7.jpg}} &
        \subfloat{\includegraphics[width = 1.5in]{figures/results/simple_template_tracker/taxi_sequence/8.jpg}} \\
       
        \subfloat{\includegraphics[width = 1.5in]{figures/results/simple_template_tracker/taxi_sequence/9.jpg}} &
        \subfloat{\includegraphics[width = 1.5in]{figures/results/simple_template_tracker/taxi_sequence/10.jpg}} &
        \subfloat{\includegraphics[width = 1.5in]{figures/results/simple_template_tracker/taxi_sequence/11.jpg}} &
        \subfloat{\includegraphics[width = 1.5in]{figures/results/simple_template_tracker/taxi_sequence/12.jpg}} \\
       
        \subfloat{\includegraphics[width = 1.5in]{figures/results/simple_template_tracker/taxi_sequence/13.jpg}} &
        \subfloat{\includegraphics[width = 1.5in]{figures/results/simple_template_tracker/taxi_sequence/14.jpg}} &
        \subfloat{\includegraphics[width = 1.5in]{figures/results/simple_template_tracker/taxi_sequence/15.jpg}} &
        \subfloat{\includegraphics[width = 1.5in]{figures/results/simple_template_tracker/taxi_sequence/16.jpg}} \\   

        \end{tabular}}
    \caption{Simple Template Tracking Hamburg Sequence}
\end{figure}

In Figure~\ref{fig:simple_template_tracking} it can be seen that the Simple Template Matching Algorithm manages to track
the car up until frame 12 of the sequence, beyond which the rotation of the car
drives the Sum of Square Differences similarity measure between the initial template chosen in
$\mathbf{f}_0$ and the region of interest in $\mathbf{f}_{12}$ below the
threshold of 0.8.

Lowering the detection threshold allows the tracker to follow the
car for a larger amount of frames, however this is not a solution to the problem
as the tracker becomes more susceptible to noise, and is certainly not a good
model for robustness across different sequences.

An idea to get around the changing template, is the implementation of an
Adaptive Template Matching Algorithm.

\subsection{Adaptive Template Matching}
As described in Section~\ref{theoretical_framework_adaptive_tm}, the assumption
that an object maintains the same appearance is only valid for an interval of
$K$ frames. Beyond this interval, $\mathbf{f}_{k}$ is sufficiently different
from $\mathbf{f}_{k+K}$ for our similarity measure to fall below the similarity
threshold, $\tau$.

The idea behind this variant of Template Tracker is that by updating the
template which we are trying to match before the we traverse more than
$K$-frames.

The sequence in Figure~\ref{fig:adaptive_template_tracking} is generated by
updating the template every frame, the located object in frame,
$\mathbf{f}_{k-1}$. becomes the template for $\mathbf{f}_k$.

\begin{figure} \label{fig:adaptive_template_tracking}
    \makebox[\linewidth][c]{
    \begin{tabular}{cccc}
        \subfloat{\includegraphics[width = 1.5in]{figures/results/adaptive_template_tracker/taxi_sequence/1.jpg}} &
        \subfloat{\includegraphics[width = 1.5in]{figures/results/adaptive_template_tracker/taxi_sequence/2.jpg}} &
        \subfloat{\includegraphics[width = 1.5in]{figures/results/adaptive_template_tracker/taxi_sequence/3.jpg}} &
        \subfloat{\includegraphics[width = 1.5in]{figures/results/adaptive_template_tracker/taxi_sequence/4.jpg}} \\

        \subfloat{\includegraphics[width = 1.5in]{figures/results/adaptive_template_tracker/taxi_sequence/5.jpg}} &
        \subfloat{\includegraphics[width = 1.5in]{figures/results/adaptive_template_tracker/taxi_sequence/6.jpg}} &
        \subfloat{\includegraphics[width = 1.5in]{figures/results/adaptive_template_tracker/taxi_sequence/7.jpg}} &
        \subfloat{\includegraphics[width = 1.5in]{figures/results/adaptive_template_tracker/taxi_sequence/8.jpg}} \\
       
        \subfloat{\includegraphics[width = 1.5in]{figures/results/adaptive_template_tracker/taxi_sequence/9.jpg}} &
        \subfloat{\includegraphics[width = 1.5in]{figures/results/adaptive_template_tracker/taxi_sequence/10.jpg}} &
        \subfloat{\includegraphics[width = 1.5in]{figures/results/adaptive_template_tracker/taxi_sequence/11.jpg}} &
        \subfloat{\includegraphics[width = 1.5in]{figures/results/adaptive_template_tracker/taxi_sequence/12.jpg}} \\
       
        \subfloat{\includegraphics[width = 1.5in]{figures/results/adaptive_template_tracker/taxi_sequence/13.jpg}} &
        \subfloat{\includegraphics[width = 1.5in]{figures/results/adaptive_template_tracker/taxi_sequence/14.jpg}} &
        \subfloat{\includegraphics[width = 1.5in]{figures/results/adaptive_template_tracker/taxi_sequence/15.jpg}} &
        \subfloat{\includegraphics[width = 1.5in]{figures/results/adaptive_template_tracker/taxi_sequence/16.jpg}} \\   

        \end{tabular}}
    \caption{Adaptive Template Tracking Hamburg Sequence}
\end{figure}

From Figure~\ref{fig:adaptive_template_tracking} it shows that the Adaptive
Template Tracker manages to locate the car within a bounding box in each frame
of the sequence. It should however be noted that there is a noticeable ``drift'' in
the localisation. 
This is due to the fact that the template selected does not solely 


\section{Colour Co-occurrence Histogram}

\section{Mean Shift Tracking}
This Section deals with the qualitative analysis of the Mean Shift Tracker
implemented in Section~\ref{implementation_mean_shift_tracker}. The analysis is
broken down into a set of experiments, each of which is based on an image
sequence that exhibits one or more of the challenges to motion tracking outlined
in Section~\ref{literature_review_challenges}. 

The image sequences used are taken from the datasets provided by the VOT2017
challenge \cite{VOT_TPAMI}.

The parameters that are within the control of the User of the MOT System Mean
Shift implementation are.
\begin{itemize}
    \item kernel size 
    \item step threshold, $\epsilon$
    \item number of histogram bins, $m$
\end{itemize}

We subsequently assess the performance of the Mean Shift Tracker implementation
against variations of these three parameters on a diverse set of sequences.
\subsection{Experiment: Effect of bin histogram bin count on Mean Shift Tracker Performance}
\subsubsection{Sequence 1: Fish}
This sequence is one of a several fish, of which a distinctly yellow fish is
tracked. The challenges to motion tracking that arise in this image sequence are the following.
\begin{itemize}
    \item Occlusion (complete)
    \item Scaling 
    \item Ego motion (slight) 
\end{itemize}

The relevant sub-sequences of interest are documented below:

\paragraph{Partial Occlusion}

\begin{figure} \label{fig:mean_shift_partial_occlusion}
    \makebox[\linewidth][c]{
    \begin{tabular}{cccc}
        \subfloat{\includegraphics[width = 1.5in]{figures/results/mean_shift_tracker/fish3/occlusion/1.jpg}} &
        \subfloat{\includegraphics[width = 1.5in]{figures/results/mean_shift_tracker/fish3/occlusion/2.jpg}} &
        \subfloat{\includegraphics[width = 1.5in]{figures/results/mean_shift_tracker/fish3/occlusion/3.jpg}} &
        \subfloat{\includegraphics[width = 1.5in]{figures/results/mean_shift_tracker/fish3/occlusion/4.jpg}} \\

        \subfloat{\includegraphics[width = 1.5in]{figures/results/mean_shift_tracker/fish3/occlusion/5.jpg}} &
        \subfloat{\includegraphics[width = 1.5in]{figures/results/mean_shift_tracker/fish3/occlusion/6.jpg}} &
        \subfloat{\includegraphics[width = 1.5in]{figures/results/mean_shift_tracker/fish3/occlusion/7.jpg}} &
        \subfloat{\includegraphics[width = 1.5in]{figures/results/mean_shift_tracker/fish3/occlusion/8.jpg}} \\
       
        \subfloat{\includegraphics[width = 1.5in]{figures/results/mean_shift_tracker/fish3/occlusion/9.jpg}} &
        \subfloat{\includegraphics[width = 1.5in]{figures/results/mean_shift_tracker/fish3/occlusion/10.jpg}} &
        \subfloat{\includegraphics[width = 1.5in]{figures/results/mean_shift_tracker/fish3/occlusion/11.jpg}} &
        \subfloat{\includegraphics[width = 1.5in]{figures/results/mean_shift_tracker/fish3/occlusion/12.jpg}} \\
       
        \subfloat{\includegraphics[width = 1.5in]{figures/results/mean_shift_tracker/fish3/occlusion/13.jpg}} &
        \subfloat{\includegraphics[width = 1.5in]{figures/results/mean_shift_tracker/fish3/occlusion/14.jpg}} &
        \subfloat{\includegraphics[width = 1.5in]{figures/results/mean_shift_tracker/fish3/occlusion/15.jpg}} &
        \subfloat{\includegraphics[width = 1.5in]{figures/results/mean_shift_tracker/fish3/occlusion/16.jpg}} \\   

        \end{tabular}}
    \caption{Challenge: Partial Occlusion}
\end{figure}

\paragraph{Changing Orientation}

\begin{figure} \label{fig:mean_shift_orientation}
    \makebox[\linewidth][c]{
    \begin{tabular}{cccc}
        \subfloat{\includegraphics[width = 1.5in]{figures/results/mean_shift_tracker/fish3/orientation/1.jpg}} &
        \subfloat{\includegraphics[width = 1.5in]{figures/results/mean_shift_tracker/fish3/orientation/2.jpg}} &
        \subfloat{\includegraphics[width = 1.5in]{figures/results/mean_shift_tracker/fish3/orientation/3.jpg}} &
        \subfloat{\includegraphics[width = 1.5in]{figures/results/mean_shift_tracker/fish3/orientation/4.jpg}} \\

        \subfloat{\includegraphics[width = 1.5in]{figures/results/mean_shift_tracker/fish3/orientation/5.jpg}} &
        \subfloat{\includegraphics[width = 1.5in]{figures/results/mean_shift_tracker/fish3/orientation/6.jpg}} &
        \subfloat{\includegraphics[width = 1.5in]{figures/results/mean_shift_tracker/fish3/orientation/7.jpg}} &
        \subfloat{\includegraphics[width = 1.5in]{figures/results/mean_shift_tracker/fish3/orientation/8.jpg}} \\
       
        \subfloat{\includegraphics[width = 1.5in]{figures/results/mean_shift_tracker/fish3/orientation/9.jpg}} &
        \subfloat{\includegraphics[width = 1.5in]{figures/results/mean_shift_tracker/fish3/orientation/10.jpg}} &
        \subfloat{\includegraphics[width = 1.5in]{figures/results/mean_shift_tracker/fish3/orientation/11.jpg}} &
        \subfloat{\includegraphics[width = 1.5in]{figures/results/mean_shift_tracker/fish3/orientation/12.jpg}} \\
       
        \subfloat{\includegraphics[width = 1.5in]{figures/results/mean_shift_tracker/fish3/orientation/13.jpg}} &
        \subfloat{\includegraphics[width = 1.5in]{figures/results/mean_shift_tracker/fish3/orientation/14.jpg}} &
        \subfloat{\includegraphics[width = 1.5in]{figures/results/mean_shift_tracker/fish3/orientation/15.jpg}} &
        \subfloat{\includegraphics[width = 1.5in]{figures/results/mean_shift_tracker/fish3/orientation/16.jpg}} \\   

        \end{tabular}}
    \caption{Challenge: Changing Orientation}
\end{figure}

\subsubsection{Sequence 2: Girl}
This sequence is one of a Girl in a park riding around on a scooter. She is
wearing a distinctive colourful outfit relative to the rest of the moving objects in the
scene, which consist mostly other humans.
The relevant challenges presented in this scene are:
\begin{itemize}
    \item Occlusion (complete)
    \item Scaling 
    \item Ego motion (slight) 
\end{itemize}

\paragraph{Complete Occlusion}
The relevant subsequence exhibiting occlusion ranges from $\mathbf{f}_{97}$ to
$\mathbf{f}_{127}$ and is presented in Figure~\ref{fig:mean_shift_complete_occlusion}.

\begin{figure}    
    \makebox[\linewidth][c]{
    \begin{tabular}{cccc}
        \subfloat{\includegraphics[width = 1.5in]{figures/results/mean_shift_tracker/girl/occlusion/1.jpg}} &
        \subfloat{\includegraphics[width = 1.5in]{figures/results/mean_shift_tracker/girl/occlusion/2.jpg}} &
        \subfloat{\includegraphics[width = 1.5in]{figures/results/mean_shift_tracker/girl/occlusion/3.jpg}} &
        \subfloat{\includegraphics[width = 1.5in]{figures/results/mean_shift_tracker/girl/occlusion/4.jpg}} \\

        \subfloat{\includegraphics[width = 1.5in]{figures/results/mean_shift_tracker/girl/occlusion/5.jpg}} &
        \subfloat{\includegraphics[width = 1.5in]{figures/results/mean_shift_tracker/girl/occlusion/6.jpg}} &
        \subfloat{\includegraphics[width = 1.5in]{figures/results/mean_shift_tracker/girl/occlusion/7.jpg}} &
        \subfloat{\includegraphics[width = 1.5in]{figures/results/mean_shift_tracker/girl/occlusion/8.jpg}} \\
       
        \subfloat{\includegraphics[width = 1.5in]{figures/results/mean_shift_tracker/girl/occlusion/9.jpg}} &
        \subfloat{\includegraphics[width = 1.5in]{figures/results/mean_shift_tracker/girl/occlusion/10.jpg}} &
        \subfloat{\includegraphics[width = 1.5in]{figures/results/mean_shift_tracker/girl/occlusion/11.jpg}} &
        \subfloat{\includegraphics[width = 1.5in]{figures/results/mean_shift_tracker/girl/occlusion/12.jpg}} \\
       
        \subfloat{\includegraphics[width = 1.5in]{figures/results/mean_shift_tracker/girl/occlusion/13.jpg}} &
        \subfloat{\includegraphics[width = 1.5in]{figures/results/mean_shift_tracker/girl/occlusion/14.jpg}} &
        \subfloat{\includegraphics[width = 1.5in]{figures/results/mean_shift_tracker/girl/occlusion/15.jpg}} &
        \subfloat{\includegraphics[width = 1.5in]{figures/results/mean_shift_tracker/girl/occlusion/16.jpg}} \\   

        \end{tabular}
    }
    \caption{Challenge: Complete Occlusion}
\end{figure}

\paragraph{Scaling}

\begin{figure} \label{fig:mean_shift_girl_scale}
    \makebox[\linewidth][c]{
    \begin{tabular}{cccc}
        \subfloat{\includegraphics[width = 1.5in]{figures/results/mean_shift_tracker/girl/scale/1.jpg}} &
        \subfloat{\includegraphics[width = 1.5in]{figures/results/mean_shift_tracker/girl/scale/2.jpg}} &
        \subfloat{\includegraphics[width = 1.5in]{figures/results/mean_shift_tracker/girl/scale/3.jpg}} &
        \subfloat{\includegraphics[width = 1.5in]{figures/results/mean_shift_tracker/girl/scale/4.jpg}} \\

        \subfloat{\includegraphics[width = 1.5in]{figures/results/mean_shift_tracker/girl/scale/5.jpg}} &
        \subfloat{\includegraphics[width = 1.5in]{figures/results/mean_shift_tracker/girl/scale/6.jpg}} &
        \subfloat{\includegraphics[width = 1.5in]{figures/results/mean_shift_tracker/girl/scale/7.jpg}} &
        \subfloat{\includegraphics[width = 1.5in]{figures/results/mean_shift_tracker/girl/scale/8.jpg}} \\
       
        \subfloat{\includegraphics[width = 1.5in]{figures/results/mean_shift_tracker/girl/scale/9.jpg}} &
        \subfloat{\includegraphics[width = 1.5in]{figures/results/mean_shift_tracker/girl/scale/10.jpg}} &
        \subfloat{\includegraphics[width = 1.5in]{figures/results/mean_shift_tracker/girl/scale/11.jpg}} &
        \subfloat{\includegraphics[width = 1.5in]{figures/results/mean_shift_tracker/girl/scale/12.jpg}} \\
       
        \subfloat{\includegraphics[width = 1.5in]{figures/results/mean_shift_tracker/girl/scale/13.jpg}} &
        \subfloat{\includegraphics[width = 1.5in]{figures/results/mean_shift_tracker/girl/scale/14.jpg}} &
        \subfloat{\includegraphics[width = 1.5in]{figures/results/mean_shift_tracker/girl/scale/15.jpg}} &
        \subfloat{\includegraphics[width = 1.5in]{figures/results/mean_shift_tracker/girl/scale/16.jpg}} \\   
    \end{tabular}}
    \caption{Challenge: Scale Change}
\end{figure}


\subsubsection{Sequence 3: Ants}
This image sequence is of multiple ants in motion within a Petri dish.

The challenges presented by this sequence are the following:
\begin{itemize}
    \item Target Speed
    \item Track Overlap
\end{itemize}

\begin{figure} \label{fig:mean_shift_girl_scale}
    \makebox[\linewidth][c]{
    \begin{tabular}{cccc}
        \subfloat{\includegraphics[width = 1.5in]{figures/results/mean_shift_tracker/ants/speed/1.jpg}} &
        \subfloat{\includegraphics[width = 1.5in]{figures/results/mean_shift_tracker/ants/speed/2.jpg}} &
        \subfloat{\includegraphics[width = 1.5in]{figures/results/mean_shift_tracker/ants/speed/3.jpg}} &
        \subfloat{\includegraphics[width = 1.5in]{figures/results/mean_shift_tracker/ants/speed/4.jpg}} \\

        \subfloat{\includegraphics[width = 1.5in]{figures/results/mean_shift_tracker/ants/speed/5.jpg}} &
        \subfloat{\includegraphics[width = 1.5in]{figures/results/mean_shift_tracker/ants/speed/6.jpg}} &
        \subfloat{\includegraphics[width = 1.5in]{figures/results/mean_shift_tracker/ants/speed/7.jpg}} &
        \subfloat{\includegraphics[width = 1.5in]{figures/results/mean_shift_tracker/ants/speed/8.jpg}} \\
       
        \subfloat{\includegraphics[width = 1.5in]{figures/results/mean_shift_tracker/ants/speed/9.jpg}} &
        \subfloat{\includegraphics[width = 1.5in]{figures/results/mean_shift_tracker/ants/speed/10.jpg}} &
        \subfloat{\includegraphics[width = 1.5in]{figures/results/mean_shift_tracker/ants/speed/11.jpg}} &
        \subfloat{\includegraphics[width = 1.5in]{figures/results/mean_shift_tracker/ants/speed/12.jpg}} \\
       
        \subfloat{\includegraphics[width = 1.5in]{figures/results/mean_shift_tracker/ants/speed/13.jpg}} &
        \subfloat{\includegraphics[width = 1.5in]{figures/results/mean_shift_tracker/ants/speed/14.jpg}} &
        \subfloat{\includegraphics[width = 1.5in]{figures/results/mean_shift_tracker/ants/speed/15.jpg}} &
        \subfloat{\includegraphics[width = 1.5in]{figures/results/mean_shift_tracker/ants/speed/16.jpg}} \\   
    \end{tabular}}
    \caption{Challenge: Target Speed}
\end{figure}





