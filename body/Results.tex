\chapter{Results}\label{chapter_results}

This Chapter is concerned with the presentation results obtained from a practical
application of the algorithms implemented in
Chapter~\ref{chapter_implementation} on various image sequences. 
The image sequences are mainly taken from the Visual Object Tracking 2017
Challenge (VOT2017)~\cite{VOT_TPAMI}, and the famous Hamburg taxi sequence is also used as an
initial experiment for the template matching trackers.

The analysis of the algorithms is both quantitative and qualitative as defined
in Section~\ref{methodology_testing}. The
quantitative analysis is based on the tracker performance per the centroid error metric
The qualitative analysis is based on both the metric and experts of bounding
boxes, it aims to provide an insight into the results based on the data informed by
an understanding of the various algorithm implementations. 

This chapter also presents the results of the overall integrated MT System, and
gauges it's performance by relating it back to the system
s user requirements stated in Section~\ref{introduction_user_requirements}.


\section{Template Matching Tracker}
This Section presents with a qualitative analysis of both the simple and
adaptive template matching trackers proposed in
Section~\ref{theoretical_framework_template_matching_trackers}. 

Both trackers were applied to the famous Hamburg taxi sequence \cite{} as an initial
test of their applicability to the motion tracking problem.
The template trackers below were run for a threshold $\tau=0.8$.

\subsection{Simple template matching tracker}\label{results_simple_template_matching}
The initial template used is the shown by the rectangular region around the car
in the first frame. The images shown correspond to almost equally spaced
samples of the 40 frame taxi sequence.

\begin{figure}\label{fig:simple_template_tracking}    
    \makebox[\linewidth][c]{
    \begin{tabular}{cccc}
        \subfloat{\includegraphics[width = 1.5in]{figures/results/simple_template_tracker/taxi_sequence/1.jpg}} &
        \subfloat{\includegraphics[width = 1.5in]{figures/results/simple_template_tracker/taxi_sequence/2.jpg}} &
        \subfloat{\includegraphics[width = 1.5in]{figures/results/simple_template_tracker/taxi_sequence/3.jpg}} &
        \subfloat{\includegraphics[width = 1.5in]{figures/results/simple_template_tracker/taxi_sequence/4.jpg}} \\

        \subfloat{\includegraphics[width = 1.5in]{figures/results/simple_template_tracker/taxi_sequence/5.jpg}} &
        \subfloat{\includegraphics[width = 1.5in]{figures/results/simple_template_tracker/taxi_sequence/6.jpg}} &
        \subfloat{\includegraphics[width = 1.5in]{figures/results/simple_template_tracker/taxi_sequence/7.jpg}} &
        \subfloat{\includegraphics[width = 1.5in]{figures/results/simple_template_tracker/taxi_sequence/8.jpg}} \\
       
        \subfloat{\includegraphics[width = 1.5in]{figures/results/simple_template_tracker/taxi_sequence/9.jpg}} &
        \subfloat{\includegraphics[width = 1.5in]{figures/results/simple_template_tracker/taxi_sequence/10.jpg}} &
        \subfloat{\includegraphics[width = 1.5in]{figures/results/simple_template_tracker/taxi_sequence/11.jpg}} &
        \subfloat{\includegraphics[width = 1.5in]{figures/results/simple_template_tracker/taxi_sequence/12.jpg}} \\
       
        \subfloat{\includegraphics[width = 1.5in]{figures/results/simple_template_tracker/taxi_sequence/13.jpg}} &
        \subfloat{\includegraphics[width = 1.5in]{figures/results/simple_template_tracker/taxi_sequence/14.jpg}} &
        \subfloat{\includegraphics[width = 1.5in]{figures/results/simple_template_tracker/taxi_sequence/15.jpg}} &
        \subfloat{\includegraphics[width = 1.5in]{figures/results/simple_template_tracker/taxi_sequence/16.jpg}} \\   

        \end{tabular}}
    \caption{Simple template tracker applied to Hamburg taxi sequence}
\end{figure}

In Figure~\ref{fig:simple_template_tracking} it can be seen that the Simple
template tracker (STT) manages to track the car up until $\mathbf{F}_{6}$
in the Figure, corresponding to $\mathbf{f}_{12}$ in the sequence, beyond which
the rotation of the car drives the sum of square differences similarity measure
between the initial template chosen in $\mathbf{f}_0$ and the region of interest
in $\mathbf{f}_{12}$ below the threshold.

It is important to highlight that, while lowering the detection threshold allows
the tracker to follow the car for a larger amount of frames, this is not a
solution to the problem as the tracker becomes more susceptible to noise, and is
certainly not a good model for robustness across different sequences.

An idea to get around the changing template, is the implementation of an
adaptive template matching algorithm, in which we update the model we are
looking for each $\mathbf{f}_k$.

\subsection{Adaptive template matching tracker}\label{results_adaptive_template_matching}
As described in Section~\ref{theoretical_framework_adaptive_tm}, the assumption
that an object maintains the same appearance is only valid for an interval of
$K$ frames. Beyond this interval, $\mathbf{f}_{k}$ is sufficiently different
from $\mathbf{f}_{k+K}$ for our similarity measure to fall below the similarity
threshold, $\tau$. For the experiments a threshold of $\tau=0.8$ was used.

The idea behind the adaptive template tracker (ATT) is that we update the
template which we are trying to match before the we traverse more than
$K$ frames.

\subsubsection{Hamburg taxi sequence}
The sequence in Figure~\ref{fig:adaptive_template_tracking} is generated by
updating the template every frame, the located object in frame,
$\mathbf{f}_{k-1}$. becomes the template for $\mathbf{f}_k$.

\begin{figure}     
    \makebox[\linewidth][c]{
    \begin{tabular}{cccc}
        \subfloat{\includegraphics[width = 1.5in]{figures/results/adaptive_template_tracker/taxi_sequence/1.jpg}} &
        \subfloat{\includegraphics[width = 1.5in]{figures/results/adaptive_template_tracker/taxi_sequence/2.jpg}} &
        \subfloat{\includegraphics[width = 1.5in]{figures/results/adaptive_template_tracker/taxi_sequence/3.jpg}} &
        \subfloat{\includegraphics[width = 1.5in]{figures/results/adaptive_template_tracker/taxi_sequence/4.jpg}} \\

        \subfloat{\includegraphics[width = 1.5in]{figures/results/adaptive_template_tracker/taxi_sequence/5.jpg}} &
        \subfloat{\includegraphics[width = 1.5in]{figures/results/adaptive_template_tracker/taxi_sequence/6.jpg}} &
        \subfloat{\includegraphics[width = 1.5in]{figures/results/adaptive_template_tracker/taxi_sequence/7.jpg}} &
        \subfloat{\includegraphics[width = 1.5in]{figures/results/adaptive_template_tracker/taxi_sequence/8.jpg}} \\
       
        \subfloat{\includegraphics[width = 1.5in]{figures/results/adaptive_template_tracker/taxi_sequence/9.jpg}} &
        \subfloat{\includegraphics[width = 1.5in]{figures/results/adaptive_template_tracker/taxi_sequence/10.jpg}} &
        \subfloat{\includegraphics[width = 1.5in]{figures/results/adaptive_template_tracker/taxi_sequence/11.jpg}} &
        \subfloat{\includegraphics[width = 1.5in]{figures/results/adaptive_template_tracker/taxi_sequence/12.jpg}} \\
       
        \subfloat{\includegraphics[width = 1.5in]{figures/results/adaptive_template_tracker/taxi_sequence/13.jpg}} &
        \subfloat{\includegraphics[width = 1.5in]{figures/results/adaptive_template_tracker/taxi_sequence/14.jpg}} &
        \subfloat{\includegraphics[width = 1.5in]{figures/results/adaptive_template_tracker/taxi_sequence/15.jpg}} &
        \subfloat{\includegraphics[width = 1.5in]{figures/results/adaptive_template_tracker/taxi_sequence/16.jpg}} \\   

        \end{tabular}}
    \caption{Adaptive template tracker applied to Hamburg taxi sequence\label{fig:adaptive_template_tracking}}
\end{figure}

From Figure~\ref{fig:adaptive_template_tracking} it shows that the ATT manages
to locate the car within a bounding box in each frame of the sequence. It should
however be noted that there is a noticeable ``drift'' in
the localisation. 
This is due to the fact that the algorithm has no way of controlling the amount of
background noise that is allowed into the template.

\subsubsection{Adaptive template tracker vs occlusion}
The graph in Figure~\ref{fig:results/adaptive_template_tracker/metric.png}, shows the
performance of the MST according to the average error metric outlined in
Section~\ref{methodology_data_analysis}. 

Figure~\ref{fig:adaptive_template_occlusion} shows the visual bounding box and
underlying template used for the data in presented in the graph. $\mathbf{F}_{13}$ to
$\mathbf{F}_{16}$ detail the point of failure. As the girl crosses in front of
the man, the adaptive template is updated with information from the man in the
background which as the girl moves away becomes the larger match, thus the
tracker loses the girl. 

\Figure[width=0.8\columnwidth]{Graph of average error of adaptive tracker (pixels) vs frames in sequence}{results/adaptive_template_tracker/metric.png}

\begin{figure}     
    \makebox[\linewidth][c]{
    \begin{tabular}{cccc}
        \subfloat{\includegraphics[width = 1.5in]{figures/results/adaptive_template_tracker/girl/2.jpg}} &
        \subfloat{\includegraphics[width = 1.5in]{figures/results/adaptive_template_tracker/girl/3.jpg}} &
        \subfloat{\includegraphics[width = 1.5in]{figures/results/adaptive_template_tracker/girl/4.jpg}} &
        \subfloat{\includegraphics[width = 1.5in]{figures/results/adaptive_template_tracker/girl/5.jpg}} \\

        \subfloat{\includegraphics[width = 1.5in]{figures/results/adaptive_template_tracker/girl/6.jpg}} &
        \subfloat{\includegraphics[width = 1.5in]{figures/results/adaptive_template_tracker/girl/7.jpg}} &
        \subfloat{\includegraphics[width = 1.5in]{figures/results/adaptive_template_tracker/girl/8.jpg}} &
        \subfloat{\includegraphics[width = 1.5in]{figures/results/adaptive_template_tracker/girl/9.jpg}} \\
       
        \subfloat{\includegraphics[width = 0.3in]{figures/results/adaptive_template_tracker/girl/10.jpg}} &
        \subfloat{\includegraphics[width = 0.3in]{figures/results/adaptive_template_tracker/girl/11.jpg}} &
        \subfloat{\includegraphics[width = 0.3in]{figures/results/adaptive_template_tracker/girl/12.jpg}} &
        \subfloat{\includegraphics[width = 0.3in]{figures/results/adaptive_template_tracker/girl/13.jpg}} \\
       
        \subfloat{\includegraphics[width = 0.3in]{figures/results/adaptive_template_tracker/girl/14.jpg}} &
        \subfloat{\includegraphics[width = 0.3in]{figures/results/adaptive_template_tracker/girl/15.jpg}} &
        \subfloat{\includegraphics[width = 0.3in]{figures/results/adaptive_template_tracker/girl/16.jpg}} &
        \subfloat{\includegraphics[width = 0.3in]{figures/results/adaptive_template_tracker/girl/17.jpg}} \\   

        \end{tabular}}
    \caption{Adaptive template tracker against occlusion\label{fig:adaptive_template_occlusion}}
\end{figure}

From the graph in Figure\ref{fig:adaptive_template_occlusion} we can
quantitative observe tracker drift (as was also the case in the Hamburg taxi
sequence) by the steady growth of the average tracker error between
$\mathbf{f}_0$ and $\mathbf{f}_{40}$ before failing completely.
Before it loses track of the girl, the tracker has an average tracking error of
6 pixels between the tracked center and that of the ground truth values.

Depending on it's similarity threshold $\tau$, the adaptive template tracker
either fails to locate the object in the case of a larger $\tau$ and rapid
occlusion between successive frames, $\mathbf{f}_k$ and $\mathbf{f}_{k+1}$ or
adds the occlusion to the target model in the case of a lower $\tau$, and a
slowly progressing occlusion between $\mathbf{f}_k$ and $\mathbf{f}_{k+1}$, as
is the case in Figure~\ref{fig:adaptive_template_occlusion}, where the frames
and their corresponding target models (updated templates) are shown. 


\section{Colour Co-occurrence Histogram Detector}
This Section details the results of the Colour co-occurrence histogram detector
(CCH-detector) implemented in Section~\ref{implementation_ch}.
As discussed in Section~\ref{theoretical_framework_ch}, the rationale behind
implementing the CCH-detector is assessing whether it could overcome the
shortcomings of the template matching approaches in terms of drift due to not
being able to filter out the background, and lack of generalisation.

\subsection{Colour co-occurrence histogram against occlusion}
The adaptive template tracker failed immediately when faced with the challenge
of occlusion in the girl sequence. We apply the CCH-detector to a scenario within
the girl sequence to see whether it can successfully overcome the challenge of
occlusion.

The experiment is performed as follows; The model of the girl is extracted in
$\mathbf{f}_0$ of the sequence. This model is then used to localize
the girl in another frame, $\mathbf{f}_{138}$ in which she is partially
occluded.

The localization is performed by a two level search as described in
Section~\ref{implementation_ch}, with parameters $n_c=8$ and $n_d=12$ which
\cite{Chang1999} rigorously proved to be the optimal values to minimize the false
alarm probability. (see Section~\ref{theoretical_framework_ch}).

Figure~\ref{fig:ch_partial_occlusion} details the results, the green bounding
box is the outcome of the initial coarse grained search of $\mathbf{f}_{138}$.
The result of the fine grained search within the green box region is highlighted
by the yellow bounding box. 
The CCH-detector effectively detects the girl within the $\mathbf{f}_{138}$ in
spite of significant occlusion on by the man obscuring the girl. 

\begin{figure}     
    \makebox[\linewidth][c]{
    \begin{tabular}{cc}
        \subfloat{\includegraphics[width = 2.5in]{figures/results/ch_detector/results_girl/frame.jpg}} &
        \subfloat{\includegraphics[width = 2.5in]{figures/results/ch_detector/results_girl/result.jpg}} \\

        \end{tabular}}
    \caption{Colour co-occurrence histogram detection against occlusion\label{fig:ch_partial_occlusion}
    }
\end{figure}


\section{Mean Shift Tracker}
This Section deals with the analysis of the mean shift tracker (MST)
implemented in Section~\ref{implementation_mean_shift_tracker}. The analysis is
broken down into a set of experiments each dealing with a different image
sequence of interest.

There are generally three parameters that a user can vary when using the MST
\begin{itemize}
    \item $\epsilon$ - step size bounding maximum mean shift vector magnitude
    \item $m$ - bin count of histograms
    \item $(h_x,h_y)$ - kernel dimensions and positioning around or on object
\end{itemize}
We subsequently assess the performance of the MST implementation
for varying parameters $\epsilon$ and $m$ for the sequence in order to see the
effect their on tracker performance, the goal being to select reasonable values
for the two parameters to asses the general practical performance of the Mean
Shift Tracker. 

We perform this assessment in a series of experiments, each of which is based on an image
sequence that exhibits one or more of the challenges to motion tracking outlined
in Section~\ref{literature_review_challenges}. 
 
\subsection{Effect of step size, $\epsilon$}\label{results_eps}
The effect of step size, $\epsilon$ is quantified easily, but it can be reasoned
about. In Listing~\ref{lst:loop}, the step size is the initial check for
termination of a single mean shift iteration (locating an object). As mentioned
before, the $\epsilon$ value corresponds to how large our mean shift is allowed
to be, if a large mean shift vector magnitude is permissible, more loops will
complete without the iterative refinement step, this in turn means faster execution time.
However, it also means that the bounding box follows are less smooth
trajectory between frames for larger values of $\epsilon$.


\subsection{Effect of bin count, $m$}\label{results_m}
This experiment aims to quantitatively determine the effect of varying the
bin count, $m$ on the execution time of calculating the histogram, and computing
the Bhattacharyya coefficient.

The setup is simple enough, the execution time of the \textit{get\_pdf()} and
\textit{get\_BC()} functions was averaged over 1000 iterations for varying
values of m for a template of dimensions (100,100).

The effect of varying $m$ on histogram generation is shown in
Figure~\ref{fig:results_m_pdf}. There is not significant trend shown between the
bin count and the execution time of the get\_pdf() function. This makes sense
because we still iterate over all the pixels to allocate each of them to a
particular bin. The access time of the bins in the underlying numpy array is not
a significant bottle neck for array sizes (bin counts) ranging 1 to 256.

The effect of varying $m$ on the calculation of the Bhattacharyya coefficient is
detailed in Figure~\ref{fig:results_m_bc}. The execution time of the get\_BC() goes  

\Figure[width=0.7\columnwidth]{Graph showing effect of bin size (m) on execution time (ms) of \textit{get\_pdf()}}{results/mean_shift_tracker/results_m_pdf}

\Figure[width=0.7\columnwidth]{Graph showing effect of bin size (m) on execution time (ms) of \textit{get\_BC()}}{results/mean_shift_tracker/results_m_bc}

The experiments 1-3 aim to assess performance of the MST in dealing with the
various motion tracking challenges, for varying sizes of kernel initialisation.
The general idea is to compare the performance of a large kernel (yellow)
encompassing the whole object of interest, and a smaller kernel (green), that is
initialised around some feature of the larger object. 

For the subsequent experiments, the parameters $\epsilon$ and $m$ of the MST are
kept constant as $\epsilon=5$ and $m=8$. These choices were informed by the results in
Sections~\ref{results_eps} and~\ref{results_m}. 

\subsection{Experiment 1: fish}
This sequence is one in which there are several fish, of which a distinctly yellow fish is
tracked. The challenges to motion tracking that arise in this image sequence are the following.
\begin{itemize}
    \item occlusion/track overlap
    \item Scaling 
\end{itemize}

The graph in Figure~\ref{fig:results/mean_shift_tracker/metric.png} shows the
performance of the MST according to the average error metric outlined in
Section~\ref{methodology_data_analysis}. 
The motion tracking challenges present within the sequence are highlighted
between vertical bars, with the partial occlusion/track overlap challenge and the changing orientation challenge occurring in the green domain.
Reference to the graph is made in the treatment of the discussion of the
particular challenges that follows.

\Figure[width=0.8\columnwidth]{Graph of average error of MST (pixels) vs frames in sequence}{results/mean_shift_tracker/fish3/metric.png}

\subsubsection{Partial occlusion}\label{mean_shift_partial_occlusion}
The relevant subsequence exhibiting partial occlusion ranges from
$\mathbf{f}_{178}$ to $\mathbf{f}_{240}$ of the original sequence. 

This interval is represented by the orange domain of the graph in
Figure~\ref{fig:results/mean_shift_tracker/fish3/metric.png}. From the graph it can be
seen that the mean shift tracker actually performs better under occlusion than in
some other sub sequences that exhibit motion, the average error between the
tracker centroid and the ground truth centroids in this region is around 5
pixels, part of which could also be attributed to different initialisations for
the fish tracker and the regions defining the ground truths.

In Figure~\ref{fig:mean_shift_partial_occlusion} Our target, the yellow fish
remains stationary from $\mathbf{F_{1}}$ to $\mathbf{F_{16}}$. Both kernels are
unaffected by the second gray fish, occupying the same pixel space as our
target. 

Both the metric and the image frames suggest that the MST is relatively robust
to partial occlusion and track overlap in the fish sequence. This is due to the fact that the target
fish and the second fish are easily distinguishable within the RGB colour-space
from which we derive our histograms. Therefore, the tracker is not drawn to the
second fish. As we still partially see a large part of our target when
occluded, our similarity, $\rho$ at the should remain greatest at the position
of the target despite the partial occlusion, which seems to be the case as the
tracker does not drift.

\begin{figure}     
    \makebox[\linewidth][c]{
    \begin{tabular}{cccc}
        \subfloat{\includegraphics[width = 1.5in]{figures/results/mean_shift_tracker/fish3/occlusion/1.jpg}} &
        \subfloat{\includegraphics[width = 1.5in]{figures/results/mean_shift_tracker/fish3/occlusion/2.jpg}} &
        \subfloat{\includegraphics[width = 1.5in]{figures/results/mean_shift_tracker/fish3/occlusion/3.jpg}} &
        \subfloat{\includegraphics[width = 1.5in]{figures/results/mean_shift_tracker/fish3/occlusion/4.jpg}} \\

        \subfloat{\includegraphics[width = 1.5in]{figures/results/mean_shift_tracker/fish3/occlusion/5.jpg}} &
        \subfloat{\includegraphics[width = 1.5in]{figures/results/mean_shift_tracker/fish3/occlusion/6.jpg}} &
        \subfloat{\includegraphics[width = 1.5in]{figures/results/mean_shift_tracker/fish3/occlusion/7.jpg}} &
        \subfloat{\includegraphics[width = 1.5in]{figures/results/mean_shift_tracker/fish3/occlusion/8.jpg}} \\
       
        \subfloat{\includegraphics[width = 1.5in]{figures/results/mean_shift_tracker/fish3/occlusion/9.jpg}} &
        \subfloat{\includegraphics[width = 1.5in]{figures/results/mean_shift_tracker/fish3/occlusion/10.jpg}} &
        \subfloat{\includegraphics[width = 1.5in]{figures/results/mean_shift_tracker/fish3/occlusion/11.jpg}} &
        \subfloat{\includegraphics[width = 1.5in]{figures/results/mean_shift_tracker/fish3/occlusion/12.jpg}} \\
       
        \subfloat{\includegraphics[width = 1.5in]{figures/results/mean_shift_tracker/fish3/occlusion/13.jpg}} &
        \subfloat{\includegraphics[width = 1.5in]{figures/results/mean_shift_tracker/fish3/occlusion/14.jpg}} &
        \subfloat{\includegraphics[width = 1.5in]{figures/results/mean_shift_tracker/fish3/occlusion/15.jpg}} &
        \subfloat{\includegraphics[width = 1.5in]{figures/results/mean_shift_tracker/fish3/occlusion/16.jpg}} \\   

        \end{tabular}}
    \caption{Challenge: partial occlusion\label{fig:mean_shift_partial_occlusion}
 }
\end{figure}

\subsubsection{Changing orientation}
The relevant subsequence exhibiting the challenge of ranges from
$\mathbf{f}_{430}$ to $\mathbf{f}_{512}$ of the fish sequence.  

This interval is represented by the green domain of the graph in
Figure~\ref{fig:results/mean_shift_tracker/fish3/metric.png}. From the graph it can be
seen that the mean shift tracker actually performs better when faced with this
challenge than it does in the sub sequences that exhibit motion, the average error between the
tracker centroid and the ground truth centroids in this region is around 4
pixels. 

\begin{figure}     
    \makebox[\linewidth][c]{
    \begin{tabular}{cccc}
        \subfloat{\includegraphics[width = 1.5in]{figures/results/mean_shift_tracker/fish3/orientation/1.jpg}} &
        \subfloat{\includegraphics[width = 1.5in]{figures/results/mean_shift_tracker/fish3/orientation/2.jpg}} &
        \subfloat{\includegraphics[width = 1.5in]{figures/results/mean_shift_tracker/fish3/orientation/3.jpg}} &
        \subfloat{\includegraphics[width = 1.5in]{figures/results/mean_shift_tracker/fish3/orientation/4.jpg}} \\

        \subfloat{\includegraphics[width = 1.5in]{figures/results/mean_shift_tracker/fish3/orientation/5.jpg}} &
        \subfloat{\includegraphics[width = 1.5in]{figures/results/mean_shift_tracker/fish3/orientation/6.jpg}} &
        \subfloat{\includegraphics[width = 1.5in]{figures/results/mean_shift_tracker/fish3/orientation/7.jpg}} &
        \subfloat{\includegraphics[width = 1.5in]{figures/results/mean_shift_tracker/fish3/orientation/8.jpg}} \\
       
        \subfloat{\includegraphics[width = 1.5in]{figures/results/mean_shift_tracker/fish3/orientation/9.jpg}} &
        \subfloat{\includegraphics[width = 1.5in]{figures/results/mean_shift_tracker/fish3/orientation/10.jpg}} &
        \subfloat{\includegraphics[width = 1.5in]{figures/results/mean_shift_tracker/fish3/orientation/11.jpg}} &
        \subfloat{\includegraphics[width = 1.5in]{figures/results/mean_shift_tracker/fish3/orientation/12.jpg}} \\
       
        \subfloat{\includegraphics[width = 1.5in]{figures/results/mean_shift_tracker/fish3/orientation/13.jpg}} &
        \subfloat{\includegraphics[width = 1.5in]{figures/results/mean_shift_tracker/fish3/orientation/14.jpg}} &
        \subfloat{\includegraphics[width = 1.5in]{figures/results/mean_shift_tracker/fish3/orientation/15.jpg}} &
        \subfloat{\includegraphics[width = 1.5in]{figures/results/mean_shift_tracker/fish3/orientation/16.jpg}} \\   

        \end{tabular}}
    \caption{Challenge: changing orientation\label{fig:mean_shift_orientation}
 }
\end{figure}

Experiment 1 shows that the MST is relatively robust to both partial occlusion and changing
object orientation for objects that are easily distinguishable from their
backgrounds backgrounds and occlusions within the chosen low-level feature space.

\subsection{Experiment 2: girl}
This sequence is one of a girl in a park riding around on a scooter. She is
wearing a distinctive colourful outfit relative to the rest of the moving objects in the
scene, which consist mostly other humans.
The relevant challenges presented in this scene are:
\begin{itemize}
    \item Occlusion (complete)
    \item Scaling 
    \item Ego motion  
\end{itemize}

The graph in Figure~\ref{fig:results/mean_shift_tracker/girl/metric.png} shows the
performance of the MST according to the average error metric outlined in
Section~\ref{methodology_data_analysis}. 
The motion tracking challenges present within the sequence are highlighted
between vertical bars.

\Figure[width=0.8\columnwidth]{Graph of average error of MST (pixels) vs frames in sequence}{results/mean_shift_tracker/girl/metric.png}

\subsubsection{Complete occlusion}
The relevant subsequence exhibiting occlusion ranges from $\mathbf{f}_{97}$ to
$\mathbf{f}_{127}$ in the original sequence. 

The interval exhibiting the complete occlusion / track overlap challenge occurs in
the orange domain of the graph in Figure~\ref{fig:results/mean_shift_tracker/girl/metric.png}

presented in Figure~\ref{fig:mean_shift_complete_occlusion}.

\begin{figure}    
    \makebox[\linewidth][c]{
    \begin{tabular}{cccc}
        \subfloat{\includegraphics[width = 1.5in]{figures/results/mean_shift_tracker/girl/occlusion/1.jpg}} &
        \subfloat{\includegraphics[width = 1.5in]{figures/results/mean_shift_tracker/girl/occlusion/2.jpg}} &
        \subfloat{\includegraphics[width = 1.5in]{figures/results/mean_shift_tracker/girl/occlusion/3.jpg}} &
        \subfloat{\includegraphics[width = 1.5in]{figures/results/mean_shift_tracker/girl/occlusion/4.jpg}} \\

        \subfloat{\includegraphics[width = 1.5in]{figures/results/mean_shift_tracker/girl/occlusion/5.jpg}} &
        \subfloat{\includegraphics[width = 1.5in]{figures/results/mean_shift_tracker/girl/occlusion/6.jpg}} &
        \subfloat{\includegraphics[width = 1.5in]{figures/results/mean_shift_tracker/girl/occlusion/7.jpg}} &
        \subfloat{\includegraphics[width = 1.5in]{figures/results/mean_shift_tracker/girl/occlusion/8.jpg}} \\
       
        \subfloat{\includegraphics[width = 1.5in]{figures/results/mean_shift_tracker/girl/occlusion/9.jpg}} &
        \subfloat{\includegraphics[width = 1.5in]{figures/results/mean_shift_tracker/girl/occlusion/10.jpg}} &
        \subfloat{\includegraphics[width = 1.5in]{figures/results/mean_shift_tracker/girl/occlusion/11.jpg}} &
        \subfloat{\includegraphics[width = 1.5in]{figures/results/mean_shift_tracker/girl/occlusion/12.jpg}} \\
       
        \subfloat{\includegraphics[width = 1.5in]{figures/results/mean_shift_tracker/girl/occlusion/13.jpg}} &
        \subfloat{\includegraphics[width = 1.5in]{figures/results/mean_shift_tracker/girl/occlusion/14.jpg}} &
        \subfloat{\includegraphics[width = 1.5in]{figures/results/mean_shift_tracker/girl/occlusion/15.jpg}} &
        \subfloat{\includegraphics[width = 1.5in]{figures/results/mean_shift_tracker/girl/occlusion/16.jpg}} \\   

        \end{tabular}
    }
    \caption{Challenge: complete occlusion\label{fig:mean_shift_complete_occlusion}}
\end{figure}

In Figure~\ref{fig:mean_shift_complete_occlusion}, $\mathbf{F}_4$ to
$\mathbf{F}_{13}$ show that the Mean Shift Tracker is able to relocate the Girl
within after a man completely occludes her. We see that the smaller Green Kernel
is offset by man for a bit, whereby the larger Yellow Kernel remains relatively
stable. The Green Kernel behaviour can be explained by the fact that it's model,
$\hat{q}$ was defined solely around the Girl's yellow top. This is sufficiently
close for the Man's white top within the RGB colour-space to cause a small drift
in the kernel which lasts only until the girl is visible again upon which the
tracker is pulled back towards the girl as she is the once again the local
maximum of our similarity measure $\rho$.

\subsubsection{Scaling}
The Challenge of Scaling is exhibited from $\mathbf{f}_{427}$ to
$\mathbf{f}_{997}$ in the original sequence, where the girl is walking away from
the camera, the result being that her dimensions within the frame are becoming
smaller.

The scaling challenge occurs within the green domain of the graph in
Figure~\ref{fig:results/mean_shift_tracker/girl/metric.png}.
Ego motion is present through out the scene as the camera follows the girl. 
While the MST tracker error fluctuates significantly between lows of close to 0
and highs of around 40 pixels, this seems to have to effect on the long-term
accuracy of the tracker over this sequence as it performs better on average that
the overall sequence tracking error of 11 pixels.

Figure~\ref{fig:mean_shift_girl_scale} shows that the girl is successfully tracked within until the
completion of the sequence by both the initialised kernels. The robustness is
likely in part due to the distinct nature of her attire when compared to the
rest of the scene. It should be noted that the smaller kernel seems to perform
better than the larger kernel, this cannot however be quantified by the metric
because ground values would be misleading if compared to an initialisation that
doesn't is not whole target object.

\begin{figure} 
    \makebox[\linewidth][c]{
    \begin{tabular}{cccc}
        \subfloat{\includegraphics[width = 1.5in]{figures/results/mean_shift_tracker/girl/scale/1.jpg}} &
        \subfloat{\includegraphics[width = 1.5in]{figures/results/mean_shift_tracker/girl/scale/2.jpg}} &
        \subfloat{\includegraphics[width = 1.5in]{figures/results/mean_shift_tracker/girl/scale/3.jpg}} &
        \subfloat{\includegraphics[width = 1.5in]{figures/results/mean_shift_tracker/girl/scale/4.jpg}} \\

        \subfloat{\includegraphics[width = 1.5in]{figures/results/mean_shift_tracker/girl/scale/5.jpg}} &
        \subfloat{\includegraphics[width = 1.5in]{figures/results/mean_shift_tracker/girl/scale/6.jpg}} &
        \subfloat{\includegraphics[width = 1.5in]{figures/results/mean_shift_tracker/girl/scale/7.jpg}} &
        \subfloat{\includegraphics[width = 1.5in]{figures/results/mean_shift_tracker/girl/scale/8.jpg}} \\
       
        \subfloat{\includegraphics[width = 1.5in]{figures/results/mean_shift_tracker/girl/scale/9.jpg}} &
        \subfloat{\includegraphics[width = 1.5in]{figures/results/mean_shift_tracker/girl/scale/10.jpg}} &
        \subfloat{\includegraphics[width = 1.5in]{figures/results/mean_shift_tracker/girl/scale/11.jpg}} &
        \subfloat{\includegraphics[width = 1.5in]{figures/results/mean_shift_tracker/girl/scale/12.jpg}} \\
       
        \subfloat{\includegraphics[width = 1.5in]{figures/results/mean_shift_tracker/girl/scale/13.jpg}} &
        \subfloat{\includegraphics[width = 1.5in]{figures/results/mean_shift_tracker/girl/scale/14.jpg}} &
        \subfloat{\includegraphics[width = 1.5in]{figures/results/mean_shift_tracker/girl/scale/15.jpg}} &
        \subfloat{\includegraphics[width = 1.5in]{figures/results/mean_shift_tracker/girl/scale/16.jpg}} \\   
    \end{tabular}}
    \caption{Challenge: scale change\label{fig:mean_shift_girl_scale}}
\end{figure}

\subsection{Experiment 3: ants}
This image sequence is of multiple ants in motion within a Petri dish. The Ants
are almost identical, with some being marked by a colourful mark on the abdomen
to likely help distinguish them in analysis.

The challenges presented by this sequence are the following:
\begin{itemize}
    \item Target speed
    \item Track overlap
\end{itemize}

\subsubsection{Target speed}\label{mean_shift_target_speed}
The relevant subsequence ranges from $\mathbf{f}_{202}$ to $\mathbf{f}_{250}$
in the original sequence. In Figure~\ref{fig:mean_shift_target_speed}, frames
$\mathbf{F}_{13}$ to $\mathbf{F}_{16}$ highlight this issue, as can be seen by
the Green Kernel failing to track the Ant.

\begin{figure}
    \makebox[\linewidth][c]{
    \begin{tabular}{cccc}
        \subfloat{\includegraphics[width = 1.5in]{figures/results/mean_shift_tracker/ants/speed/1.jpg}} &
        \subfloat{\includegraphics[width = 1.5in]{figures/results/mean_shift_tracker/ants/speed/2.jpg}} &
        \subfloat{\includegraphics[width = 1.5in]{figures/results/mean_shift_tracker/ants/speed/3.jpg}} &
        \subfloat{\includegraphics[width = 1.5in]{figures/results/mean_shift_tracker/ants/speed/4.jpg}} \\

        \subfloat{\includegraphics[width = 1.5in]{figures/results/mean_shift_tracker/ants/speed/5.jpg}} &
        \subfloat{\includegraphics[width = 1.5in]{figures/results/mean_shift_tracker/ants/speed/6.jpg}} &
        \subfloat{\includegraphics[width = 1.5in]{figures/results/mean_shift_tracker/ants/speed/7.jpg}} &
        \subfloat{\includegraphics[width = 1.5in]{figures/results/mean_shift_tracker/ants/speed/8.jpg}} \\
       
        \subfloat{\includegraphics[width = 1.5in]{figures/results/mean_shift_tracker/ants/speed/9.jpg}} &
        \subfloat{\includegraphics[width = 1.5in]{figures/results/mean_shift_tracker/ants/speed/10.jpg}} &
        \subfloat{\includegraphics[width = 1.5in]{figures/results/mean_shift_tracker/ants/speed/11.jpg}} &
        \subfloat{\includegraphics[width = 1.5in]{figures/results/mean_shift_tracker/ants/speed/12.jpg}} \\
       
        \subfloat{\includegraphics[width = 1.5in]{figures/results/mean_shift_tracker/ants/speed/13.jpg}} &
        \subfloat{\includegraphics[width = 1.5in]{figures/results/mean_shift_tracker/ants/speed/14.jpg}} &
        \subfloat{\includegraphics[width = 1.5in]{figures/results/mean_shift_tracker/ants/speed/15.jpg}} &
        \subfloat{\includegraphics[width = 1.5in]{figures/results/mean_shift_tracker/ants/speed/16.jpg}} \\   
    \end{tabular}}
    \caption{Challenge: target speed: small kernel loses ant\label{fig:mean_shift_ant_speed}}
\end{figure}

\begin{figure}
    \makebox[\linewidth][c]{
    \begin{tabular}{cccc}
        \subfloat{\includegraphics[width = 1.5in]{figures/results/mean_shift_tracker/ants/speed2/1.jpg}} &
        \subfloat{\includegraphics[width = 1.5in]{figures/results/mean_shift_tracker/ants/speed2/2.jpg}} &
        \subfloat{\includegraphics[width = 1.5in]{figures/results/mean_shift_tracker/ants/speed2/3.jpg}} &
        \subfloat{\includegraphics[width = 1.5in]{figures/results/mean_shift_tracker/ants/speed2/4.jpg}} \\
    \end{tabular}}
    \caption{Challenge: target speed: large kernel loses ant\label{fig_mean_shift_ant_speed2}}
\end{figure}

Keeping in mind that a digital video is a sampling of an analogue scene.
In order to discuss the effect of the ``speed'' of a target on tracker
performance it is necessary to interpret this speed more clearly
as the magnitude of an object's inter-frame displacement, $\delta$ - which
we define as the distance between an object's
position between two adjacent frames $\mathbf{f}_k$ and $\mathbf{f}_{k+1}$.
This is an adequate formulation because, depending on the sampling frequency $f_s$ of
a particular sequence, a fast object such as a car sampled at a high $f_s$ can
have a smaller $\delta$ than a relatively slow snail sampled at a lower $f_s$. 

As the $\delta$ of an object between $\mathbf{f}_k$ and $\mathbf{f}_{k+1}$ 
increases in magnitude, less of the object lies within the dimensions
$(h_x,h_y)$ of the kernel that we initialise at it's last known pixel location
$\mathbf{c_0}$ in $\mathbf{f}_k$. In terms of the mean shift tracking algorithm this simply
results in a larger mean shift vector, $\vec{m}$.

As shown in the experiment in Section~\ref{results_varying_epsilon}, The MST
parameter, $\epsilon$ - which limits the allowed magnitude of $\vec{m}$ between
$\mathbf{f}_{k}$ and $\mathbf{f}_{k+1}$ - can increase the tracker's tolerance
to challenges such as Object speed and occlusion, at the
cost of increased track ``jitter''. 

However the Green Kernel throughout Figure~\ref{fig:mean_shift_ant_speed}
highlights the extreme case in which the kernel initialised at $\mathbf{c_0}$ in
$\mathbf{f}_{k+1}$ does not even partly contain the Target.  

We do see that the larger Yellow Kernel manages to track the Ant successfully
from $\mathbf{F_1}$ to $\mathbf{F_7}$ before changing tracks for a better closer
candidate as elaborated in section~\ref{mean_shift_track_overlap}.
Figure~\ref{fig:mean_shift_ant_speed2} shows $f_{271}$ to $f_{274}$ of the original
sequence, highlighting that even the larger Yellow Kernel can fail to track the
Object for a sufficiently high inter-frame displacement.

The results in Figures~\ref{mean_shift_ant_speed} tell us
that a larger kernel larger kernel size is more robust to the effects of large
$\delta$, but as Figure~\ref{fig:mean_shift_ant_speed2}, even a
kernel enclosing the whole object of interest can lose track of the object given
a sufficiently large $\delta$. It is also not feasible to increase our kernel
size largely beyond the object dimensions, as it would incorporate a lot of
noise into the object model $\hat(q)$, likely deteriorating tracker performance
when faced with other challenges.

It is important to note that a solution to the problem can be tied back to
sampling rate, $f_s$ used to obtain the sequence. If a higher sampling rate were
used, it would mean higher redundancy between frames with smaller $\delta$
values. An impractical solution therefore could be to try and obtain the
same sequence again, but sampled at a higher $f_s$.

This is likely unfeasible. Furthermore, one can easily imagine situations in which the
available technology is constrained by factors such as budget, size etc. In such
situations, we would still like to be able to track our object of interest.  A
possible solution could be to augment the basic Mean Shift Algorithm with other
forms of more information possibly from other tracking methodologies highlighted
in Figure\ref{fig:motion_tracking_taxonomy}. 

\subsubsection{Track Overlap}\label{mean_shift_track_overlap}
In Figure~\ref{fig:mean_shift_ant_speed}, $\mathbf{F}_5$ to $\mathbf{F}_{11}$
highlight this challenge. We have two ants coming within close proximity of
each other, we can see that the yellow kernel switches to track the second Ant
after the crossing of their tracks within the sequence. Let us refer to the
original ant as $A_0$ and the second ant as $A_1$. 

It is important at this point to highlight the differences between this case of
track overlap, and that detailed in Section~\ref{mean_shift_partial_occlusion},
in which, despite the tracks for the yellow fish and grey fish overlapping
entirely, the tracker remains remains unaffected.
The answer lies within the feature space our tracker is based on. Within the RGB colour space,
the yellow and grey fish are largely distinct. Hence the tracker will not be
drawn to the grey fish as it exits the Yellow Kernel.

In this case, the are two objects within the Yellow Kernel that are practically
indistinguishable within our chose feature space. One ant exits the Kernel, there exists
no gradient within the domain of the similarity function to draw the tracker away.

With $A_0$ and $A_1$ close to each other, The fact that $A_0$ moves with a high
velocity, means that the inter-frame displacement of $A_0$ between
$\mathbf{f}_k$ and $\mathbf{f}_{k+1}$ is easily be large enough to place $A_1$
closer to $A_0$ in $\mathbf{f}_{k+1}$, which makes the mean shift iteration
likely to pick $A_1$ as the target location in $\mathbf{f}_{k+1}$

\section{Graphical User Interface}
The implemented GUI System is shown below in Figure~\ref{fig:results/gui/first_look}

\Figure[width=1\columnwidth]{Image of the MOT System GUI implementation applying
the MS-Tracker to a fish sequence}{results/gui/first_look}

\subsection{I/O functionality}
The GUI allows a user to select and 

\subsection{Algorithm Integration}
The three algorithms discussed in Section~\ref{chapter_theoretical_framework}
successfully integrated with the Front-end GUI. Their usage within the GUI is
shown in the subsequence sections.

\begin{figure}
    \makebox[\linewidth][c]{
    \begin{tabular}{ccc}
        \subfloat{\includegraphics[width = 2.2in]{figures/results/gui/CHD_1.png}} &
        \subfloat{\includegraphics[width = 2.2in]{figures/results/gui/CHD_2.png}} &
        \subfloat{\includegraphics[width = 2.2in]{figures/results/gui/CHD_3.png}} \\
    \end{tabular}}
    \caption{Algorithm User Flow\label{fig:gui_CHD}}
\end{figure}

Figure~\ref{fig:gui_CHD} outlines the user flow for performing initialising the
Co-occurrence histogram detection on a sequence.


\subsection{Parameter Selection}
In addition to implementing the algorithms, the parameters for the individual
algorithms can be varied by by the user.


