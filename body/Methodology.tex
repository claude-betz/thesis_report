\chapter{Methodology}\label{methodology}
This Chapter aims to define and develop the sequence of phases necessary for the completion
of the objectives of this project as detailed in
Section~\ref{introduction_objectives}. This chapter also outlines the process
models governing the progression through the various development phases, and
forming the approach successful completion of the project.

\section{Project Phases and Process Models }
An overview of the project phases as well as and process models is presented in
Figure~\ref{fig:methodology_phases_and_process_models}.
\Figure[width=1\columnwidth, angle=90, scale=1.3]{Block diagram of project phases and process methodologies}{methodology_phases_and_process_models}

\subsection{Project phases}
The various project phases involved in the development of the MT system are
outlined below. They correspond by number and name to the phases presented in
Figure~\ref{fig:methodology_phases_and_process_models}.

\subsubsection{Phase 1: Research and literature review}
This phase serves as an surveying of the broad field of Motion Tracking to
determine what the various established approaches to the Motion Tracking are.
The outcome of this phase is a refined scope in terms of what approaches will be
trialled to meet the User Requirements outlined in
Section~\ref{introduction_user_requirements}. We break this phase into
Chapter~\ref{chapter_literature_review} a Literature Review which broadly
surveys the field. Chapter~\ref{chapter_theoretical_framework} then breaks down
the approaches of interest in sufficient detail to allow for a ground up Design
and Implementation.

\subsubsection{Phase 2: Planning and preliminary design}
This phase involves a translation of the relevant literature and theoretical
foundation from Phase 1 into a feasible design for the MOT System. The scope of
this project requires a delineation between the Front- and Back-end
functionalities. This Phase is documented in the Design laid out in
Chapter~\ref{chapter_design}.

It is important to note that while Phase 1 and 2 are largely linear. The Nature
of Phases 3 to 5 is iterative, and make up the bulk of this project. Hence we
make the use of the Spiral Process Methodology, to ensure that at each iteration
we have a working MT System prototype.  

\subsubsection{Phase 3: Development and prototyping}
This phase involves the implementation of the planned system functionality in a
modular fashion. Given the theoretically driven approach of this project. There
is a high correlation between the theory outlined in
Chapter~\ref{chapter_theoretical_framework} and outcomes of this phase which are
presented in Chapter~\ref{chapter_implementation}.

\subsubsection{Phase 4: Experimentation and testing}
This phase follows involves a verification of the implementations on a
functional level, as well as experimentation that can be used to assess the
performance of the prototype in meeting the overall user requirements. The
outcomes of this phase feed into phase 4. 

\subsubsection{Phase 5: Data collection and analysis}
This phase is still within the iterations of the Spiral Model, it involves
analysis of the outcomes of phase 4. The insights of this feed into the 
subsequent iterations of the spiral model which iteratively refine the MT
system. 
During later iterations, the data collected and analysed in the results
presented in 
Chapter~\ref{chapter_results}, the main criteria for this being a satisfactory achievement
of the defined user requirements.

\subsubsection{Phase 6: System integration and integration testing}
This phase is concerned with the integration of front- and back-end subsystems that
are sufficiently developed. To this effect there are several considerations such
as language choice and compiler compatibility of the front- and back-end
subsystem implementations. 

\subsubsection{Phase 7: Conclusions and future plans}
This phase formalises the implications of the data collected in the various
iterations of prototyping. Tying the results back to objectives and user
requirements of the project.
It addressed possible improvements that are beyond the initial project scope
and not achievable within the project time-frame.

\section{Hierarchy of process models}
Outlined below are the various process models employed in ensuring the
successful completion of the MT system within the allocated time.

\subsection{V-diagram}
Due to the scale of the undertaking, different process models were employed to
address different aspects of the project.

The process model chosen to guide the overall progression through the phases of
this study is the V-model. This choice is tied to a number of reasons: 
\begin{itemize}
    \item It is simple to use.
    \item It provides a tight link between overall design and verification 
    \item Suited to projects with well defined user requirements.
\end{itemize}

Figure~\ref{methodology_v_diagram} details the V-Diagram process model. It
details the various levels of specification down to the implementation and
highlights the stages of testing and integration of the MT system.

\Figure[width=1\columnwidth, angle=90, scale=1.5]{V-Diagram process model}{methodology_v_diagram}


\subsection{Spiral model}
The design task which spans phases; 4,5 and 6 as shown in Figure~\ref{fig:phases_and_process_models}, is: 
\begin{itemize}
    \item Complex in nature
    \item Has a set of clearly defined requirements. 
\end{itemize}

It is for this reason that the spiral model is the chosen process model to
oversee these tasks. The focus on iterative design and prototyping allows for
convergence on a solution that can continuously be refined over the span of the
project.

\section{Data collection testing and analysis}\label{methodology_testing}
Chapter~\ref{chapter_design} details the design of the MT system in terms of
front- and back- end subsystems, this design is implemented in
Chapter~\ref{chapter_implementation}. This Section aims to outline the data
collection and analysis methods that will be used when assessing the
performance of the subsystems.

\subsection{Back-end data}
The implemented back-end, comprising of a set of trackers will be run on a
standardised data set provided by the Visual Object Tracking 2017 Challenge
(VOT2017)~\cite{VOT2017}. This data set comprises of an extensive set of image
sequences exhibiting a number of the challenges to motion tracking outlined in
Section~\ref{literature_review_challenges}. 

Analysis of these sequences will be both quantitative and qualitative, the two
approaches are subsequently elaborated on.

\subsubsection{Quantitative testing and analysis}
The VOT2017 Challenge provides annotated ground truth values for the image
sequences, these are the true bounding boxes of a target object in a particular
frame. These ground truths are what will be used to quantitatively benchmark
tracker performance, in terms of distance between the tracker localization and
the true location of the object of interest in a particular frame. This simple
metric is termed the ``tracker error''.

\subsubsection{Qualitative testing and analysis}
Despite a quantitative analysis, there are still insights to be drawn from a
qualitative inspection of the various sequences, from which results can be tied
back to the implementation choices such as low-level feature selection etc.

\subsection{Front-end}
The testing of the front-end will be performed by performance of various input
sequences to the graphical user interface (GUI). The analysis will be in assessing the
performance of the GUI in relation to the initial user requirements of the MT
system as specified in Section~\ref{introduction_user_requirements}.


