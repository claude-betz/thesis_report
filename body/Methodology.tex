\chapter{Methodology}\label{methodology}
This Chapter aims to define the sequence of phases necessary for the completion
of the objectives of this study, originally detailed in Section~\ref{introduction_objectives} of this report.
This chapter also outlines the Process Models underlying the development
approach to the successful completion of the defined Phases.

\section{Project Phases and Process Models}
An overview of the Project Phases as well as and Process Models is presented in
Figure~\ref{fig:methodology_phases_and_process_models} below:
\Figure[width=1\columnwidth, angle=90, scale=1.3]{Block Diagram of Project Phases}{methodology_phases_and_process_models}

\subsection{Project Phases}
The Project Phases are outlined below:
\begin{enumerate}
    \item Research and Literature Review
    \item Planning and Preliminary Design
    \item Development and Prototyping 
    \item Experimentation and Testing
    \item Data Collection and Analysis
    \item System Integration and Integration Testing
    \item Conclusions and Future Plans
\end{enumerate}

\subsubsection{Research and Literature Review}
This phase serves as an surveying of the broad field of Motion Tracking to
determine what the various established approaches to the Motion Tracking are.
The outcome of this phase is a refined scope in terms of what approaches will be
trialed to meet the User Requirements outlined in
Section~\ref{introduction_user_requirements}. We break this phase into
Chapter~\ref{chapter_literature_review} a Literature Review which broadly
surveys the field. Chapter~\ref{chapter_theoretical_framework} then breaks down
the approaches of interest in sufficient detail to allow for a ground up Design
and Implementation.

\subsubsection{Planning and Preliminary Design}
This phase involves a translation of the relevant literature and theoretical
foundation from Phase 1 into a feasible design for the MOT System. The scope of
this project requires a delineation between the Front- and Back-end
functionalities. This Phase is documented in the Design laid out in
Chapter~\ref{chapter_design}.

It is important to note that while Phase 1 and 2 are largely linear. The Nature
of Phases 3 to 5 is iterative, and make up the bulk of this project. Hence we
make the use of the Spiral Process Methodology, to ensure that at each iteration
we have a working MOT System prototype.  

\subsubsection{Development and Prototyping}
This phase involves the implementation of the planned system functionality in a
modular fashion. Given the theoretically driven approach of this project. There
is a high correlation between the theory outlined in
Chapter~\ref{chapter_theoretical_framework} and outcomes of this phase which are
presented in Chapter~\ref{chapter_implementation}.

\subsubsection{Experimentation and Testing}
This phase follows involves a verification of the implementations on a
functional level, as well as experimentation that can be used to assess the
performance of the prototype in meeting the overall user requirements. The
outcomes of this phase feed into phase 4. 

\subsubsection{Data Collection and Analysis}
This phase is still within the iterations of the Spiral Model, it involves
analysis of the outcomes of phase 4. The insights of this feed into the 
subsequent iterations of the spiral model which iteratively refine the MOT
System. 
During later iterations, the Data Collected and Analysed in the Results
Chapter~\ref{chapter_results}, the main criteria for this being a satisfactory achievement
of the defined user requirements.

\subsubsection{System Integration and Integration Testing}
This phase is concerned with the integration of Front- and Back-end modules that
are sufficiently developed. 

\subsubsection{Conclusions and Future Plans}
This phase formalised the implications of the data collected in the various
iterations of prototyping. Tying the results back to Objectives and User
Requirements of the undertaking.
It addressed possible improvements that are beyond the initial project scope
and not achievable within the project time-frame.



\subsection{Hierarchy of Process Models}
Outlined below are the various process models employed in ensuring the
successful completion of the MOT System within the allocated time.

\subsubsection{V-Model}
Due to the scale of the undertaking, different process models were employed to
address different aspects of the project.

The Process Model chosen to guide the overall progression through the phases of
this study is the V-Model. This choice is tied to a number of reasons: 
\begin{itemize}
    \item It is simple to use.
    \item It provides a tight link between overall design and verification 
    \item Suited to projects with well defined user requirements.
\end{itemize}

*cite design textbook*

\subsubsection{Spiral Model}
The Design Task which spans Phases; 4,5 and 6 in Figure~\ref{fig:phases_and_process_models}, is: 
\begin{itemize}
    \item Complex in nature
    \item Has a set of clearly defined requirements. 
\end{itemize}

\Figure[width=1\columnwidth, angle=90, scale=1.5]{V-Diagram Process Model}{methodology_v_diagram}



