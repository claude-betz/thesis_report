\appendix

\chapter{Theoretical Framework}

\section{K-means Clustering Algorithm}
The K-means classification algorithm is an unsupervised learning method that
attempts to find a pre- defined number of clusters, k, such that the Euclidean
distance between samples in a particular feature space is maximized between
clusters and minimized within them.


\chapter{Implementation}

\section{Co-occurrence Histograms Support functions}\label{supportCH}

\begin{lstlisting}[language=Python, caption={Additional functions for Co-occurrence Histogram Detection}, captionpos=b, label={lst:supportCH}]
def RGB_euclidean_distance(pixel):
    """computes euclidean distance between two 2D vectors"""
    return np.sqrt(pixel[0]**2+pixel[1]**2+pixel[2]**2)

def CH_vertical(q_labels2d, nc, nd):
    """defines the vertical cooccurance histogram for a specific template according to quantized colourspace labels matrix and distances"""
    # get label dims
    height, width = q_labels2d.shape[0], q_labels2d.shape[1] # get dimensions
    
    CH_y = np.zeros(shape=(nc,nc,nd-1)) # initialize zeros for CH histogram according to quantizations size
    for k_range in range(1,nd):
        for y in range(0,height-k_range):
            for x in range(0,width):
                src_index = q_labels2d[y,x] # get source index for CH_y at (y,x)
                dst_index = q_labels2d[y+k_range,x] # get destincation index for CH_y at (y+k,x)
                CH_y[src_index,dst_index,k_range-1] = CH_y[src_index,dst_index,k_range-1] + 1 # update histogram count in bin (src_index,dst_index)
                
    # make the histogram symmetric by multiplying with transpose of unidirectional cooccurance histogram
    CH_vertical = [np.sum((CH_y[:,:,k_range],CH_y[:,:,k_range].T), axis=0) for k_range in range(0,nd-1)]
    return CH_vertical

def CH_horizontal(q_labels2d, nc, nd):
    """defines the horizontal cooccurance histogram for a specific template according to quantized colourspace labels matrix and distances"""
    # get label dims
    height, width = q_labels2d.shape[0], q_labels2d.shape[1] # get dimensions
    
    CH_x = np.zeros(shape=(nc,nc,nd-1)) # initialize zeros for CH histogram according to quantizations size
    for k_range in range(1,nd):
        for y in range(0,height):
            for x in range(0,width-k_range):
                src_index = q_labels2d[y,x] # get source index for CH_y at (y,x)
                dst_index = q_labels2d[y,x+k_range] # get destincation index for CH_y at (y+k,x)
                CH_x[src_index,dst_index,k_range-1] = CH_x[src_index,dst_index,k_range-1] + 1 # update histogram count in bin (src_index,dst_index)
    
    # make the histogram symmetric by multiplying with transpose of unidirectional cooccurance histogram
    CH_horizontal = [np.sum((CH_x[:,:,k_range], CH_x[:,:,k_range].T), axis=0) for k_range in range(0,nd-1)]
    return CH_horizontal

def CH_pos_diagonal(q_labels2d, nc, nd):
    """defines the positve diagonal cooccurance histogram for a specifica template according to quantized colourspace labels"""
    # get label dims
    height, width = q_labels2d.shape[0], q_labels2d.shape[1] # get dimensions
    
    CH_pd = np.zeros(shape=(nc,nc,nd-1)) # initialize zeros for CH histogram according to quantizations size
    for k_range in range(1,nd):
        for y in range(height-1, 0+k_range-1, -1):
            for x in range(0,width-k_range):
                src_index = q_labels2d[y,x] # get source index for CH_y at (y,x)
                dst_index = q_labels2d[y-k_range,x+k_range] # get destincation index for CH_y at (y+k,x)
                CH_pd[src_index,dst_index,k_range-1] = CH_pd[src_index,dst_index,k_range-1] + 1 # update histogram count in bin (src_index,dst_index)
    
    # make the histogram symmetric by multiplying with transpose of unidirectional cooccurance histogram
    CH_pos_diag = [np.sum((CH_pd[:,:,k_range], CH_pd[:,:,k_range].T), axis=0) for k_range in range(0,nd-1)]
    return CH_pos_diag

def CH_neg_diagonal(q_labels2d, nc, nd):
    """defines the negative diagonal cooccurance histogram for a specific template according to quantized colourspace labels"""
    # get label dims
    height, width = q_labels2d.shape[0], q_labels2d.shape[1] # get dimensions
        
    CH_nd = np.zeros(shape=(nc,nc,nd-1)) # initialize zeros for CH histogram according to quantizations size
    for k_range in range(1,nd):
        for y in range(0,height-k_range):
            for x in range(0,width-k_range):
                src_index = q_labels2d[y,x] # get source index for CH_y at (y,x)
                dst_index = q_labels2d[y+k_range,x+k_range] # get destincation index for CH_y at (y+k,x)
                CH_nd[src_index,dst_index,k_range-1] = CH_nd[src_index,dst_index,k_range-1] + 1 # update histogram count in bin (src_index,dst_index)
    
    # make the histogram symmetric by multiplying with transpose of unidirectional cooccurance histogram
    CH_neg_diag = [np.sum((CH_nd[:,:,k_range], CH_nd[:,:,k_range].T), axis=0) for k_range in range(0,nd-1)]
    return CH_neg_diag  

\end{lstlisting}

\section{Mean Shift Tracker Support functions}\label{supportMS}
This Section provides the implementations of the support functions needed to
implement the function modules outlined in Chapter~\ref{chapter_implementation}
that correspond to the steps in the Mean Shift Algorithm outlined in
Figure~\ref{fig:design_mean_shift_tracking_algorithm}.

\begin{lstlisting}[language=Python, caption={Additional functions for Mean Shift Tracker Impementation}, captionpos=b, label={lst:support_MS}]
def get_elliptical_mask(height,width):
    """computes elliptical mask for particular ROI dimensions"""
    y0, x0 = height//2, width//2 # get center of kernel (treated as 0,0) - note these are also hy and hx 
    yy, xx = np.mgrid[:height, :width] # create layers for subsequent mask
    ellipse_eqn = ((yy-y0)/y0)**2 + ((xx-x0)/x0)**2 # ellipse equation
    return np.logical_and(ellipse_eqn<=1, ellipse_eqn<=1) # elliptical mask for kernel

def euclidean_distance(v1,v2):
    """computes euclidean distance between two 2D vectors"""
    return np.sqrt((v2[0]-v1[0])**2+(v2[1]-v1[1])**2)

def epanechnikov_eqn(x):
    """equation of epanechnikov kernel based on normalised euclidean norm squared"""
    d = 2 #number of dimensions
    c_d = np.pi# area of a unit circle in d  
    return 1/2*(1/c_d)*(d+2)*(1-x) # |x|<=1 i.e.normalised

def epanechnikov_kernel(template):
    """computes epanechnikov kernel values given distance from center of kernel"""    
    # get dimensions and generate mask
    height, width = (template.shape[0],template.shape[1])
    y0, x0 = (height//2, width//2) # get center of kernel (treated as 0,0) - note these are also hy and hx 
    elliptical_mask = get_elliptical_mask(height,width) 
    
    # get epanechnikov weights for the template dimensions
    kernel = np.zeros(shape=(height,width)) 
    for y in range(0,height):
        for x in range(0,width):
            if(elliptical_mask[y,x]==True):
                y_s, x_s  = (y-y0)/y0, (x-x0)/x0 # compute normalized point 
                norm = euclidean_distance((y_s,x_s),(0,0)) # get norm of normalized point (x_i*) i.e y0=hy, x0=hx
                kernel[y,x] = epanechnikov_eqn(norm**2) # apply norm^2 to epanechnikov equation
    C = 1/np.sum(kernel)
    return kernel,C
\end{lstlisting}
\newpage
\section{GUI Code}
\begin{lstlisting}[language=Python, caption={QLabel Mouse Event Code}, captionpos=b, label={lst:gui}]
"""
    Claude Betz (BTZCLA001)
    imageView.py

    Subclasses the QLabel class to add drag and drop functionality for a User to
    select a region of interest.
"""

from PyQt5.QtWidgets import QWidget, QLabel, QRubberBand
from PyQt5.QtGui import QImage, QPixmap
from PyQt5.QtCore import Qt, QRect, QSize, pyqtSignal

class imageView(QLabel):
    """image view subclasses QLabel to render and add mouse events"""
    currentQRect = 0
    triggered = pyqtSignal() # to send
    def __init__(self):
        super(imageView, self).__init__()
        self.pixmap = QPixmap() # to store image
        self.cropLabel = QLabel() # user selected crop

    # Mouse Events
    def mousePressEvent(self, eventQMouseEvent):
        self.originQPoint = eventQMouseEvent.pos()
        self.currentQRubberBand = QRubberBand(QRubberBand.Rectangle, self)
        self.currentQRubberBand.setGeometry(QRect(self.originQPoint, QSize()))
        self.currentQRubberBand.show()

    def mouseMoveEvent(self, eventQMouseEvent):
        self.currentQRubberBand.setGeometry(QRect(self.originQPoint, eventQMouseEvent.pos()).normalized())

    def mouseReleaseEvent(self, eventQMouseEvent):
        #self.currentQRubberBand.hide()
        QRect = self.currentQRubberBand.geometry()
        self.currentQRect = (QRect.y(), QRect.x(), QRect.height(), QRect.width())
        #print(self.currentQRect)
        self.currentQRubberBand.deleteLater()
        cropPixmap = self.pixmap.copy(QRect)
        self.cropLabel.setPixmap(cropPixmap)
        # cropPixmap.save('output.png')
        self.triggered.emit()
\end{lstlisting}


\begin{lstlisting}[language=Python, caption={GUI Impementation Code}, captionpos=b, label={lst:gui}]
"""
    Claude Betz (BTZCLA001)
    app.py

    Implements the GUI of the MOT System
"""

#import sys
import os, re, os.path # directory management
import time # threads

# PyQt5 imports
from PyQt5.QtWidgets import QMainWindow, QApplication, QWidget, QAction,
QWidgetAction, qApp, QFileDialog, QVBoxLayout, QHBoxLayout, QLabel, QPushButton,
QStyle, QMessageBox 
from PyQt5.QtGui import QIcon, QImage, QPixmap, QPainter,
QPen
from PyQt5.QtCore import Qt, QObject, QRect, pyqtSlot, pyqtSignal, QTimer 

# cv for reading
import cv2 as cv

# own classes
from front_end.imageview import imageView

# Backend imports 
from back_end.trackerTM import trackerTM # template tracker 
from back_end.detectorCH import detectorCH # ch detector
from back_end.trackerMS import trackerMS # mean shift tracker


class GUI(QMainWindow):
    # instance variables for loading a sequence
    seq_dir = "" # path to sequence folder
    seq_paths = [] # paths to all the files in sequence
    seq_length = 0 # length of sequence loaded
    cur_index = 0 # index for sequence paths
    cur_path = "" # current image path 
    cur_img = None # current image 
    
    # instance variables for playing 
    frame_delay = 0.1 # frame delay (seconds)
    seq_play = False # truth value to inform whether to play or not
    
    # instance variables for algorithms
    alg_running = False # set to true to render bounding box
    mst_delay = 0.01

    # writing 
    write_path = "" # path to directory to write to 

    # boolean Flags
    CH_model_exists = False

    def __init__(self, parent=None):
        """super constructor + app constructor"""
        super(GUI, self).__init__(parent) # Super constructor
      
        # Set window dimensions, title and icon
        self.setGeometry(50,50, 500, 500)
        self.setWindowTitle('Motion Tracking Application')
        self.setWindowIcon(QIcon('resources/logo.png'))
                
        # Initialise menu
        self.menu() 

        # Initialise window components
        self.layout()

        # Initialise API
        self.API()

        # Render
        self.show()
             
    def menu(self):               
        """Menu Bar, Task Bar and Status Bar intitialisation"""     
        # ICONS
        fileIcon = QApplication.style().standardIcon(QStyle.SP_DirIcon)
        saveIcon = QApplication.style().standardIcon(QStyle.SP_DriveFDIcon)
        exitIcon = QIcon('resources/exit-2.png')
        detectIcon = QIcon('resources/radar.png')
        trackIcon = QIcon('resources/target.png')
        settingsIcon = QIcon('resources/settings-1.png')

        # ACTIONS        
        # 1. fileMenu Actions
        ## fileSelectAction: lets user select a file from a directory 
        fileSelectAction = QAction(fileIcon, 'Select Sequence', self) # Exit Action Object
        fileSelectAction.setShortcut('Ctrl+O') # bind shortcut "Ctrl+Q" to Exit Button
        fileSelectAction.setStatusTip('Open Sequence') # status bar exit message
        fileSelectAction.triggered.connect(self.openSequence) # connect QtGui quit() method 
        
        ## exitAction: exits the application
        sequenceSaveAction = QAction(saveIcon, 'Save Sequence', self) # Exit Action Object
        sequenceSaveAction.setShortcut('Ctrl+S') # bind shortcut "Ctrl+Q" to Exit Button
        sequenceSaveAction.setStatusTip('Save next tracker run') # status bar exit message
        sequenceSaveAction.triggered.connect(self.saveSequence) # connect QtGui quit() method  

        settingsAction = QAction(settingsIcon, 'Algorithm Parameters', self) # Exit Action Object
        settingsAction.setShortcut('Ctrl+P') # bind shortcut "Ctrl+Q" to Exit Button
        settingsAction.setStatusTip('Select Algorithm Parameters') # status bar exit message
        settingsAction.triggered.connect(self.algorithmSettings) # connect QtGui quit() method  


        ## exitAction: exits the application
        exitAction = QAction(exitIcon, 'Exit', self) # Exit Action Object
        exitAction.setShortcut('Ctrl+Q') # bind shortcut "Ctrl+Q" to Exit Button
        exitAction.setStatusTip('Exit Application') # status bar exit message
        exitAction.triggered.connect(self.close) # connect QtGui quit() method 
        
        # 2. trackerMenu Actions 
        # Simple Template Matching
        simpleTemplateMatchingAction = QAction(trackIcon, 'Simple Template Tracker', self)
        simpleTemplateMatchingAction.setShortcut('Ctrl+2')
        simpleTemplateMatchingAction.setStatusTip('Apply Simple Template Matching Algorithm')
        simpleTemplateMatchingAction.triggered.connect(self.simpleTemplateTracking)

        # Adaptive Template Matching
        adaptiveTemplateMatchingAction = QAction(trackIcon, 'Adaptive Template Tracker', self)
        adaptiveTemplateMatchingAction.setShortcut('Ctrl+3')
        adaptiveTemplateMatchingAction.setStatusTip('Apply Adaptive Template Matching Algorithm')
        adaptiveTemplateMatchingAction.triggered.connect(self.adaptiveTemplateTracking)

        # Mean Shift
        meanShiftTrackingAction = QAction(trackIcon, 'Mean Shift Tracker', self)
        meanShiftTrackingAction.setShortcut('Ctrl+4')
        meanShiftTrackingAction.setStatusTip('Apply Mean Shift Tracking Algorithm')
        meanShiftTrackingAction.triggered.connect(self.meanShiftTracking)

        # Cooccurence Histogram
        detectionAction = QAction(detectIcon, 'Cooccurence Histogram Detection', self)
        detectionAction.setShortcut('Ctrl+1')
        detectionAction.setStatusTip('Apply Cooccurrence histogram Detection')
        detectionAction.triggered.connect(self.CHDetection)

        # *** menuBar ***
        menuBar = self.menuBar() 

        ## 1. fileMenu - drop down buttons
        fileMenu = menuBar.addMenu('&File') # Add File option to Menu Bar
        fileMenu.addAction(fileSelectAction) # Add fileSelectAction to fileMenu 
        fileMenu.addAction(sequenceSaveAction) # Add sequenceSaveAction to fileMenu
        fileMenu.addAction(settingsAction) # Add settingsAction to fileMenu
        fileMenu.addAction(exitAction) # Add exitAction to fileMenu
        
         ## 2. detectorMenu - select algorithms
        detectorMenu = menuBar.addMenu('&Detector')
        detectorMenu.addAction(detectionAction) # Add simpleTemplateMatchingAction to trackerMenu
        
        ## 2. trackerMenu - select algorithms
        trackerMenu = menuBar.addMenu('&Tracker')
        trackerMenu.addAction(simpleTemplateMatchingAction) # Add simpleTemplateMatchingAction to trackerMenu
        trackerMenu.addAction(adaptiveTemplateMatchingAction) # Add adaptiveTemplateMatchingAction to trackerMenu
        trackerMenu.addAction(meanShiftTrackingAction) # Add meanShiftTracking to algorithmMenu
        
        # *** statusBar ***
        self.statusBar()
        self.statusBar().showMessage('Status: Ready to Track')

        # *** Tool Bar ***
        toolBar = self.addToolBar('Exit') # Add Exit icon to Tool Bar  
        
        toolBar.addAction(fileSelectAction) # Add File Select Object to Tool Bar
        toolBar.addAction(sequenceSaveAction) # Add Seqeuence Save Object to Tool Bar
        toolBar.addAction(settingsAction) # settings
        toolBar.addAction(exitAction) # exit


    def layout(self):
        """Builds the application area where backend images displayed"""
        # screen
        self.centralWidget = QWidget() # central widget for screen
        self.setCentralWidget(self.centralWidget)

        # main screen layout
        self.screenLayout = QHBoxLayout(self.centralWidget) # main screen

        self.rightWidget = QWidget() # widget for right hand side
        self.rightLayout = QVBoxLayout(self.rightWidget) # right part of screen

        # buttons for our screen
        self.buttonsWidget = QWidget() # buttons in below right part
        self.buttonsWidgetLayout = QHBoxLayout(self.buttonsWidget)
        
        # create buttons for buttonWidget
        controls = ['','', '', '', '','']
        self.buttons = [QPushButton(b) for b in controls]
        for button in self.buttons:
            self.buttonsWidgetLayout.addWidget(button)
        
        # Button Actions
        # 0. First
        self.buttons[0].setToolTip('Click to go to Start of Sequence')
        self.buttons[0].clicked.connect(self.startImage) # action for 'start' button
        self.buttons[0].setIcon(QApplication.style().standardIcon(QStyle.SP_MediaSkipBackward))

        # 1. Prev
        self.buttons[1].setToolTip('Click to go to Previous Frame')
        self.buttons[1].clicked.connect(self.prevImage)
        self.buttons[1].setIcon(QApplication.style().standardIcon(QStyle.SP_MediaSeekBackward))

        # 2. Pause
        self.buttons[2].setToolTip('Click to Pause Sequence')
        self.buttons[2].clicked.connect(self.pauseSequence)
        self.buttons[2].setIcon(QApplication.style().standardIcon(QStyle.SP_MediaPause))

        # 3. Play
        self.buttons[3].setToolTip('Click to Play Sequence')
        self.buttons[3].clicked.connect(self.playSequence)
        self.buttons[3].setIcon(QApplication.style().standardIcon(QStyle.SP_MediaPlay))

        # 4. Next
        self.buttons[4].setToolTip('Click to go to Next Frame')
        self.buttons[4].clicked.connect(self.nextImage) # action for 'next' button
        self.buttons[4].setIcon(QApplication.style().standardIcon(QStyle.SP_MediaSeekForward))

        # 5. End
        self.buttons[5].setToolTip('Click to go to end of Sequence')
        self.buttons[5].clicked.connect(self.endImage) # action for 'end' button
        self.buttons[5].setIcon(QApplication.style().standardIcon(QStyle.SP_MediaSkipForward))

        # where to display our Main Image
        self.imageLabel = imageView()
        self.imageLabel.setMinimumWidth(480)
        self.imageLabel.setMinimumHeight(500)
        self.imageLabel.setStyleSheet('* {background: gray;}')
        self.imageLabel.setAlignment(Qt.AlignCenter)
        
        # where to display templates and buttons
        self.templateLabel = imageView()
        self.templateLabel.setMinimumWidth(240)
        self.templateLabel.setMinimumHeight(250)
        self.templateLabel.setStyleSheet('* {background: black;}')
        self.templateLabel.setAlignment(Qt.AlignCenter)

        # add widgets to layout
        self.screenLayout.addWidget(self.imageLabel)
        self.rightLayout.addWidget(self.templateLabel)
        self.rightLayout.addWidget(self.buttonsWidget)
        self.screenLayout.addWidget(self.rightWidget)
        
    def API(self):
        """initialised our API"""
        self.MST = trackerMS(self.seq_dir+"00000001.jpg") # initialise trackerMS object
        self.TMT = trackerTM(self.seq_dir+"00000001.jpg") # initialise trackerTM object
        self.CHD = detectorCH(self.seq_dir+"00000001.jpg") # intialise detectorCH object
                
    # function for sequence Input and Output
    @pyqtSlot()
    def released(self):
        print("relleeeeeeasadef")

    @pyqtSlot()
    def openSequence(self):
        """function to point to a folder with a desired sequence"""
        self.seq_dir = QFileDialog.getExistingDirectory(self,'Open File') # store path to sequence
        self.readSequence(self.seq_dir) # read in file names in sequence into self.paths
  
    def readSequence(self, seq_dir):
        """read in sequence"""
        for path in sorted(os.listdir(seq_dir)):
            full_path = os.path.join(seq_dir, path)
            if os.path.isfile(full_path):
                self.seq_paths.append(full_path)
        
        self.seq_length = len(self.seq_paths) # update sequence length
        self.cur_index = 0 # reset image index
        self.cur_path = self.seq_paths[self.cur_index] # set cur image to first image in self.seq_paths
        self.loadImage() # load in first image
  
    @pyqtSlot()
    def saveSequence(self):
        """function to select dir for next algorithm run to be saved to"""
        

    @pyqtSlot()
    def algorithmSettings(self):
        """set the algortihms with buttons"""

    @pyqtSlot()
    def loadImage(self):
        """function to display using filepath"""
        self.cur_img = cv.imread(self.cur_path) # use cv to load in self.cur_img
        qformat = QImage.Format_Indexed8 # define format of image

        if(len(self.cur_img.shape)==3): # format is [0,1,2] [rows,cols,components]
            if(self.cur_img.shape[2]==4): # check dimensions
                qformat = QImage.Format_RGBA8888
            else:
                qformat = QImage.Format_RGB888
        img = QImage(self.cur_img, self.cur_img.shape[1], self.cur_img.shape[0], self.cur_img.strides[0], qformat)
        img = img.rgbSwapped() # convert bgr to rgb

        self.cur_pixmap = QPixmap.fromImage(img) # update our internal current pixmap
        self.renderImage() # render

    def renderImage(self):
        """renders the image to the QLabel"""
        if(self.alg_running == True):
            self.drawBounds() # draw bounds if we are in algorithm
            self.renderTemplate() # load template into widget
        self.imageLabel.setPixmap(self.cur_pixmap) # populate label
        self.imageLabel.setMinimumHeight(self.cur_img.shape[0])
        self.imageLabel.setMinimumWidth(self.cur_img.shape[1])
        
    def renderTemplate(self):
        """renders template into templateQLabel when algorithm running"""
        # extract template from cur_pixmap
        template = QRect(self.x0, self.y0, self.w+1, self.h+1)
        self.tmp_pixmap = self.cur_pixmap.copy(template)

        self.templateLabel.setPixmap(self.tmp_pixmap) # populate label
        self.templateLabel.setMinimumHeight(self.h)
        self.templateLabel.setMinimumWidth(self.w)
        
    def updateImage(self):
        """updates the image being displayed in self.imageLabel"""
        self.loadImage() # load updated image

    # functions to navigate sequences
    @pyqtSlot()
    def startImage(self):
        """function to go to start of loaded sequence"""
        self.cur_index = 0
        self.cur_path = self.seq_paths[self.cur_index] # update internal current image path
        self.updateImage() # update our imageLabel

    @pyqtSlot()
    def endImage(self):
        """function to go to end of loaded sequence"""
        self.cur_index = self.seq_length - 1
        self.cur_path = self.seq_paths[self.cur_index] # update internal current image path
        self.updateImage() # update our imageLabel

    @pyqtSlot()
    def nextImage(self):
        """button for moving to next image in self.seq_dir"""
        if(self.cur_index < self.seq_length - 1):
            self.cur_index = self.cur_index + 1 # increment to next img
        else: # wrap around
            self.cur_index = 0

        self.cur_path = self.seq_paths[self.cur_index] # update internal current image path
        self.updateImage() # update our imageLabel

    @pyqtSlot()
    def prevImage(self):
        """button for movin to previous image in self.seq_dir"""
        if(self.cur_index > 0):
            self.cur_index = self.cur_index - 1 # decrement to prev image
        else: # wrap around
            self.cur_index = self.seq_length - 1
        
        self.cur_path = self.seq_paths[self.cur_index] # update internal current image path
        self.updateImage()

    def play(self):
        """gets next image while """
        if(self.seq_play==True):
            self.nextImage() # get next Image 
        else:
            self.seq_timer.stop() # stop playing
    
    @pyqtSlot()
    def pauseSequence(self):
        """pauses loaded sequence for manual iteration"""
        self.seq_play = False # don't play 

    @pyqtSlot()
    def playSequence(self):
        """plays loaded Sequence at specified playback rate""" 
        self.seq_play = True # play sequence
        self.seq_timer = QTimer() # timere for frame rate
        self.seq_timer.timeout.connect(self.play) # connect timeouts to fetching next image
        self.seq_timer.start(int(self.frame_delay * 1000)) # set timer countdown rate

    def drawBounds(self):
        """draws bounds on the appropriate object according to backend coordinates"""
        painter = QPainter(self.cur_pixmap)
        pen = QPen(Qt.yellow)
        painter.setPen(pen)
        painter.drawRect(self.x0, self.y0, self.w, self.h)

    # BACKEND functions 
    def algorithm_reset(self):
        """reset parameters after an algorithm has been run"""
        self.alg_running = False # set flag false again
        self.statusBar().setStatusTip("Status: Ready to Track")
        self.imageLabel.triggered.disconnect() # disconnect slot
        self.CH_model_exists = False # say no model

    @pyqtSlot()
    def simpleTemplateTracking(self):
        """perform simple template matching"""
        self.statusBar().showMessage('Status: Tracking (Simple Template)') 
        self.imageLabel.triggered.connect(self.simpleInit)
        QMessageBox.about(self, "Simple Template Tracker", "Select Object to Track") 
            
    
    def simpleInit(self):
        """simple specific init"""
        self.TMT.setup(self.cur_img, self.imageLabel.currentQRect, "simple") # setup simple tracking
        self.templateTrackingInit()

    @pyqtSlot()
    def adaptiveTemplateTracking(self):
        """perform adaptive template matching"""    
        self.statusBar().showMessage('Status: Tracking (Adaptive Template)')
        QMessageBox.about(self, "Adaptive Template Tracker", "Select Object to Track") 
            
        self.imageLabel.triggered.connect(self.adaptiveInit) 

    def adaptiveInit(self):
        """adaptive specific init"""
        self.TMT.setup(self.cur_img, self.imageLabel.currentQRect,"adaptive") # setup adaptive tracking
        self.templateTrackingInit() # call tracking

    def templateTrackingInit(self):
        """simple TMT initialisation"""
        self.alg_running = True # tell system and Algorithm is running
        self.TM_timer = QTimer() # timer for frame rate
        self.TM_timer.timeout.connect(self.templateTrackingLoop) # connect timeouts to fetching next image
        self.TM_timer.start(int(self.mst_delay * 1000)) # set timer countdown rate        
        self.y0 = self.imageLabel.currentQRect[0]
        self.x0 = self.imageLabel.currentQRect[1]
        self.h = self.imageLabel.currentQRect[2]
        self.w = self.imageLabel.currentQRect[3]
        self.nextImage() # load frame 1

    def templateTrackingLoop(self):
        """TMT Loop"""
        if(self.cur_index<self.seq_length-1):
            coords = self.TMT.track(self.cur_img) # track    
            print("coords: ", coords)
            self.y0 = coords[0]
            self.x0 = coords[1]
            self.nextImage() # load image    
        else:
            self.TM_timer.stop() # terminate algorithm
            self.algorithm_reset() # cleanup and reset
    
    @pyqtSlot()
    def CHDetection(self):
        self.statusBar().showMessage('Status: Detecting (Co-occurrence Histogram)')
        if(self.CH_model_exists==False):
            self.CH_model_exists==True
            QMessageBox.about(self, "Co-occurrence Histogram Detector", "Select Object to Detect") 
            self.imageLabel.triggered.connect(self.CHDInit) # select model
        else:
            self.CHDdetect() # detect to use old model

    def CHDInit(self):
        """initialise CH Histogram"""
        self.alg_running = True # tell system and Algorithm is running
        self.CHD.setup(self.cur_img, self.imageLabel.currentQRect) # setup mean shift tracker with coords
        self.y0 = self.imageLabel.currentQRect[0]
        self.x0 = self.imageLabel.currentQRect[1]
        self.h = self.imageLabel.currentQRect[2]
        self.w = self.imageLabel.currentQRect[3]
        self.CHDdetect()

    def CHDdetect(self):
        """detect template"""
        coords1, coords2 = self.CHD.detect(self.cur_img) # detect template 
        y0, x0 = coords1[0], coords1[1]
        y1, x1 = coords2[0], coords2[1]

        painter = QPainter(self.cur_pixmap)
        pen = QPen(Qt.yellow)
        painter.setPen(pen)
        painter.drawRect(x0, y0, self.w*2, self.h*2)
        painter.drawRect(x1, y1, self.w, self.h)

        self.imageLabel.setPixmap(self.cur_pixmap) # populate label
        self.imageLabel.setMinimumHeight(self.cur_img.shape[0])
        self.imageLabel.setMinimumWidth(self.cur_img.shape[1])
        
        # render template
        self.x0 = x0
        self.y0 = y0
        self.h = self.h*2
        self.w = self.w*2
        self.renderTemplate()
        self.algorithm_reset()

    @pyqtSlot()
    def meanShiftTracking(self):
        self.statusBar().showMessage('Status: Tracking (Mean Shift)')
        QMessageBox.about(self, "Mean Shift Tracking", "Select Object to Track")
        self.imageLabel.triggered.connect(self.meanShiftSelection)

    def meanShiftSelection(self):
        self.meanShiftTrackingInit()
        self.MST_timer = QTimer() # timer for frame rate
        self.MST_timer.timeout.connect(self.meanShiftTrackingLoop) # connect timeouts to fetching next image
        self.MST_timer.start(int(self.mst_delay * 1000)) # set timer countdown rate

    def meanShiftTrackingInit(self):
        """initialise meanShiftTracking"""
        self.alg_running = True # tell system and Algorithm is running
        print(self.imageLabel.currentQRect)
        self.MST.setup(self.cur_img, self.imageLabel.currentQRect) # setup mean shift tracker with coords
        self.y0 = self.imageLabel.currentQRect[0]
        self.x0 = self.imageLabel.currentQRect[1]
        self.h = self.imageLabel.currentQRect[2]
        self.w = self.imageLabel.currentQRect[3]
        self.nextImage() # load frame 1

    def meanShiftTrackingLoop(self):
        """implement loop"""   
        if(self.cur_index<self.seq_length-1):
            coords = self.MST.track(self.cur_img) # track    
            print("coords: ", coords)
            self.y0 = coords[0]
            self.x0 = coords[1]
            self.nextImage() # load image    
        else:
            self.MST_timer.stop() # terminate algorithm
            self.algorithm_reset() # cleanup and reset
\end{lstlisting}


