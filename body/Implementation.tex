\chapter{Implementation} \label{chapter_implementation}

This Chapter details the implementation of the relevant software modules of the
Overall MOT System. The work is firmly grounded within the Theoretical Framework
established in Chapter~\ref{chapter_theoretical_framework} and the System Design
developed in Chapter~\ref{chapter_design}. The Implementations documented here
are subsequently tested in Chapter~\ref{chapter_results}.

\section{Mean Shift Tracker Implementation}\label{implementation_mean_shift_tracker}
The implementation follows directly from the General Algorithm description is
Section~\ref{theoretical_framework_mean_shift_algorithm}.

This Section begins by laying out some preliminaries choices for the algorithm
implementation, then details the implementation of the various functions and
their overall integration into a working Mean Shift Tracker. The  

\subsection{Preliminaries}
This Section details implementation choices for the Mean Shift Tracker:
\begin{itemize}
    \item Feature Space
    \item  
\end{itemize}

\subsection{Computing target/candidate pdfs}
As outlined in the class structure in the UML diagram, %Figure~\ref{design_UML}, 
computing the pdf according to Equation~\ref{eqn:mean_shift_histogram} requires the following functions to be
defined

\begin{itemize}
    \item get\_ROI(frame)
    \item get\_elliptical\_mask(height, width)
    \item get\_euclidean\_distance(x1, x2):
    \item get\_epanechnikov\_weight(x):
    \item get\_epanechnikov\_kernel(ROI):
    \item get\_histogram(template, m\_bins): 
\end{itemize}

The above function implementations are provided in the following code snippet.

\begin{lstlisting}[language=Python]
def get_elliptical_mask(height, width):
    """computes elliptical mask for particular ROI dimensions"""
    y0, x0 = height//2, width//2 # get center of kernel (treated as 0,0) - note these are also hy and hx 
    yy, xx = np.mgrid[:height, :width] # create layers for subsequent mask
    ellipse_eqn = ((yy-y0)/y0)**2 + ((xx-x0)/x0)**2 # ellipse equation
    return np.logical_and(ellipse_eqn<=1, ellipse_eqn<=1) # elliptical mask for kernel

def euclidean_distance(v1,v2):
    """computes the euclidean distance between two 2D vectors"""
    return np.sqrt((v2[0]-v1[0])**2+(v2[1]-v1[1])**2)

def epanechnikov_eqn(x):
    """equation of epanechnikov kernel based on normalised euclidean norm squared"""
    d = 2 #number of dimensions
    c_d = np.pi # area of a unit circle in d  
    return 1/2*(1/c_d)*(d+2)*(1-x) # |x|<=1 i.e. normalised
\end{lstlisting}

\subsection{Computing Bhattacharyyaa Coefficient}
Computing the Bhattacharyyaa coefficient between two pdfs is implemented by the
following function according to Equation~\ref{eqn:bhattacharyya}
\begin{itemize}
    \item get\_BC(q,p)
\end{itemize}

\subsection{Computing pixel weights}
Computing the pixel weights within the region of interest requires the following
functions according to Equation~\ref{eqn:mean_shift_weights}.

\begin{itemize}
    \item get\_elliptical\_mask(height, width)
    \item get\_pixel\_weights(ROI, q, p)
\end{itemize}

\subsection{Computing mean shift vector}
Computing of the Mean Shift Vector is implemented the following functions
according to 
%Equation~\ref{eqn:mean_shift_vector}.

\begin{itemize}
    \item get\_elliptical\_mask(ROI)
    \item get\_mean\_shift\_vector(ROI, pixel\_weights)
\end{itemize}

\begin{lstlisting}

\end{lstlisting}

\subsection{Performing the Mean Shift Loop}
The Mean Shift Loop is outlined by the Algorithm in
Figure~\ref{fig:mean_shift_tracking_algorithm}, the mean shift loop is
applied to a frame $\mathbf{f}\_{k+1}$ given the model pdf of the object of
interest $\hat{q}$ and its location $\mathbf{c_0}$ in the previous frame
$\mathbf{f}_{k}$, to compute it's current position $\mathbf{c_1}$ in
$\mathbf{f}_{k+1}$

It depends on the following functions.

\begin{itemize}
    \item histogram(ROI, m\_bins)
    \item get\_pixel\_weights(ROI, q, p)
    \item euclidean\_distance(x1, x2)
    \item get\_mean\_shift\_vector(ROI, pixel\_weights)
    \item get\_BC
\end{itemize}



