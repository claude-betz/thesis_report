\chapter{Introduction}

\section{Background to study}
Motion draws the attention of an observer. A moving object in the field of
motion of an observer of a certain scene could hint towards danger or another
sort of change to said observer's immediate environment. With advancements in technology,
It is of interest to automate this task in a manner that strives for the same
level of sophistication in terms of reliability, speed, resistance to ``noise''
and the ability to generalize that humans or other animals exhibit in performing
this task.

Motion Tracking is a very application specific endeavour. Approaches that
reliably track the motion of pedestrian in a city scene may fail fall short at
detecting the motion of an animal in the woods. This is because classically,
motion tracking is a rather complex problem that, depending on approach,
requires considerable fine tuning of model parameters and assumptions before
arrival at a reliable solution to a problem.  For example, the introduction of
occlusion to moving object, or changes in illumination of a scene can all
drastically affect model performance should it not intrinsically account for
this.

\section{Objectives}\label{introduction_objectives}
\subsection{Problems to be investigated}
The problem investigated in this study is that of reliably isolating a moving
object from a video.

\subsection{Purpose of study}
The purpose of this study is the building of a piece software that can be used
to analyse scenes and allow users to isolate objects of interest within said
scenes

\section{Scope and Limitations}
This scope of is concerned with the development and testing of a Motion Tracking
program. The program will based on Classical (pre-Deep Learning) Motion Tracking
approaches and algorithms.
The project aims to primarily be able to reliably track a single object within a
given video.

\section{Outline of Thesis}

