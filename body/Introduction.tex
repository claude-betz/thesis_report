\chapter{Introduction}\label{chapter_introduction}
This Chapter provides a background to the project and motivation for project.
It outlines the main objectives of study, highlighting the scope and limitations
of the undertaking before settling on a set of user requirements for the final
product.
This chapter also provides a brief overview of all the chapters meant to help a
reader navigate the content of this thesis.

\section{Background to Study}
Motion draws the attention of an observer. A moving object in the field of
motion of an observer of a certain scene could hint towards danger or another
sort of change to said observer's immediate environment. Advancements in
technology necessitate the same level of sophistication in terms of
reliability, speed, resistance to ``noise'' and the ability to generalize that
humans or other animals exhibit in performing this task.

Motion tracking is a very application specific endeavour. For example, An
algorithm that reliably tracks the motion of pedestrian in a city scene may fall
short at detecting the motion of an animal in the woods. Even after evaluation
the trade-offs between various approach, this problem still requires
considerable fine tuning of model parameters assumptions before arrival at a
reliable solution to a problem. For example, the introduction of occlusion to a
scene, or changes in illumination of a scene can all drastically affect model
performance should it not intrinsically account for this.

\section{Objectives}\label{introduction_objectives}
This Section details the outcomes and motivation behind this research project.
Section~\ref{introduction_problem} elaborates on the nature of the investigation
while the Section~\ref{introduction_purpose} qualifies the investigation.

\subsection{Problem statement}\label{introduction_problem}
The general problem investigated in this study is that of motion tracking- The
localisation of a known object within a frame $\mathbf{f}_k$ in subsequent
frames of a given video or image sequence.  

The idea is to gain an understanding of solutions that address this problem and
select a subset of these to be trialed. We will assess the strengths and
weaknesses of the solutions when applied to a variety of situations.

From the insights during the Literature Review in Chapter~\ref{chapter_literature_review}, the
research question becomes to what extent are kernel based motion tracking
methods effective in solving the motion tracking problem?

\subsection{Purpose of study}\label{introduction_purpose}
The purpose of this study is to gain traction in the field in Computer Vision,
specifically by exploring the task of Motion Tracking. The approach taken is a
bottom up approach in which emphasis is placed on firm theoretical
understanding, which is then translated into a product.

The outcome of this endeavour is the development of a Motion Tracker
(MT) System that enables a user to analyse video sequences in which they would like
to track a particular object in motion, and isolate it's motion. The application
loosely draws inspiration from sports analysis segments where presented would often like
to highlight the motion of a player within a particular game.

\section{Scope and Limitations}\label{introduction_scope}
The scope of this Thesis includes: 
\begin{itemize}
    \item A review of literature relevant to the problem, and usage of
        said literature in the development of the Motion Tracking System.
    \item The application of relevant project management methodologies in the
        achievement of the overall system objective.
    \item Hierarchical design of the motion tracking system at various levels.
        In line with user requirements and technical specifications.
    \item The development of the system in line with good software development
        practices according to the specified design.
    \item Testing of functionality at a unit and system level
    \item Benchmarking of motion tracking system performance both at a
        functional and user level.
    \item Treatment of results, conclusions and recommendations.
\end{itemize}
This Thesis is a 12-week undertaking.

\section{User Requirements}\label{introduction_user_requirements}
The base user requirements of the proposed MT System are elaborated upon by the
following points:
\begin{itemize}
    \item The GUI application should have a menu that allows for a system
        user to select a video or image sequence of interest.
    \item Upon the selection of an image sequence, the system will allow for a
        user to specify an object or point of interest within the first sequence
        frame for tracking in subsequent frames.
    \item The system should reliably output a bounding box around the user
        specified point or object of interest in subsequent frames of the
        sequence. 
    \item The system should be allow for storage of the processed sequence to
        a user selected directory.
\end{itemize}

\section{Outline of thesis}
The main sections of this thesis are presented below  In line with the objectives and scope
of this study, 

Chapter~\ref{chapter_literature_review} provides a review of the
relevant literature. We first identify the general sub-tasks of the overall
Motion Tracking problem. Solutions to these sub-tasks are then surveyed and
assessed in line with the objectives, purpose and user requirements. 

Chapter~\ref{chapter_theoretical_framework}  



