\chapter{Introduction}\label{chapter_introduction}
This Chapter provides a background and motivation for this thesis' investigation
into the field of motion tracking. It outlines the main objectives of study,
highlighting the scope and limitations of the undertaking before settling on a
set of user requirements for the final product.  This chapter also provides a
brief overview of all the chapters, this is meant
to help a reader navigate the content of this thesis.

\section{Background to Study}
Motion draws the attention of an observer. A moving object in the field of
motion of an observer of a certain scene could hint towards danger or another
sort of change to said observer's immediate environment. Advancements in
technology aim to have machines exhibit the same level of sophistication in terms of
reliability, speed, resistance to ``noise'' and the ability to generalise that
humans or other animals exhibit in performing this task. Some notable
application areas of this include, motion capture for animation, surveillance
systems, target systems and sports analysis.

Motion tracking is a very application specific endeavour. For example, An
algorithm that reliably tracks the motion of a pedestrian in a city scene may fall
short at detecting the motion of an animal in the woods. A military targeting
system would have drastically different constraints to say a sports analyst
who highlights player movement to draw attention to a particular play.
Accordingly, there are several approaches to the problem. Even after evaluation
the trade-offs between various approaches, this problem can still require
considerable fine tuning of model parameters and assumptions before arrival at a
reliable solution to a particular application domain. 
For example, the introduction of occlusion to a scene, or changes in
illumination of a scene can all drastically affect model performance should it
not intrinsically account for this. 

\section{Objectives}\label{introduction_objectives}
This Section details the outcomes of this research project.
Subsection~\ref{introduction_problem} states the research problem to be
investigated while the Subsection~\ref{introduction_purpose} outlines the
purpose of the investigation.

\subsection{Problem statement}\label{introduction_problem}
The general problem investigated in this study is that of motion tracking- The
localisation of a known object within a frame $\mathbf{f}_k$ in subsequent
frames of a given video or image sequence.  
The goal is to gain an understanding of solutions that address this problem and
select a subset of these to be trialled. We will assess the strengths and
weaknesses of the solutions when applied to a variety of situations.

From the insights during the Literature Review in Chapter~\ref{chapter_literature_review}, the
research question becomes is defined to deal with the extent to which kernel based motion tracking
methods effective in solving the motion tracking problem?

\subsection{Purpose of study}\label{introduction_purpose}
The purpose of this study is to gain traction in the field in Computer Vision,
specifically by exploring the task of Motion Tracking. The approach taken is a
bottom up approach in which emphasis is placed on firm theoretical
understanding, which is then translated into a product.

Drawing inspiration from sport analysis segments. The outcome of this endeavour
is the development of a Motion Tracker (MT) System that enables a user to
analyse video sequences in which they would like to track a particular object in
motion, and isolate it's motion. The application loosely draws inspiration from
sports analysis segments where presented would often like to highlight the
motion of a player within a particular game.

\section{Scope and Limitations}\label{introduction_scope}
The scope of this Thesis includes: 
\begin{itemize}
    \item A review of literature relevant to the motion tracking problem, and usage of
        said literature in the development of the MT System.
    \item The application of relevant project management methodologies in the
        achievement of the overall system objective.
    \item Hierarchical design of the motion tracking system at various levels.
        In line with well defined user requirements.
    \item The development of the MT System in line with good software development
        practices according and in accordance with the specified methodology and design.
    \item Testing of the system functionality at a unit and system level.
    \item Benchmarking of MT System performance both qualitatively and
        quantitatively at the system and subsystem level.
    \item A comprehensive presentation of the results, conclusions and areas of
        future work 
\end{itemize}
This thesis is a 12-week undertaking.

\section{User Requirements}\label{introduction_user_requirements}
The base user requirements of the proposed MT System are elaborated upon by the
following points:
\begin{itemize}
    \item A GUI application should have a menu that allows for a system
        user to select a video or image sequence of interest.
    \item Upon the selection of an image sequence, the system will allow for a
        user to specify an object or point of interest within the first sequence
        frame for tracking in subsequent frames.
    \item The system should reliably output a bounding box around the user
        specified point or object of interest in subsequent frames of the
        sequence. 
    \item The system should be allow for storage of the processed sequence to
        a user selected directory.
\end{itemize}

\section{Outline of thesis}
This Section aims to present and summarise the main Chapters of this thesis
(excluding this introduction Chapter). The chapter summaries are presented below, and
are intended to act as a road map to the structure of the project.

\subsection{Literature Review}
Chapter~\ref{chapter_literature_review} provides a review of the
relevant literature. It begins by identifying the general sub-tasks of the
overall motion tracking problem. Solutions to these sub-tasks are then surveyed and
feed into the refinement of the objectives, purpose and user requirements of
this study.

\subsection{Methodology}
Chapter~\ref{chapter_methodology} aims to state and elaborate on the various
project phases involved in the development and successful completion of this
project. It settles on a set of governing process model that are necessary to
effectively govern the progression through the project phases. It also outlines
the data collection and analysis framework employed in the presentation of the
results in Chapter~\ref{chapter_results}.

\subsection{Theoretical Framework}
Chapter~\ref{chapter_theoretical_framework} is an extensive survey into the
relevant theory necessary to implement the MT system. It forms the foundation of
the subsequent system design and implementation. This is in line with the objective of
theoretically driven design and implementation. 

\subsection{Design}
Chapter~\ref{chapter_design} is an extensive development of the
two subsystems of the overall MT system and the approach to their eventual
integration. It serves as scaffolding for the iterative implementation in project
phases 3- 5 governed by the spiral model.

\subsection{Implementation}
Chapter~\ref{chapter_implementation} details the
implementation of the system in line with the design according to the spiral
model. The presentation follows closely the progression of the theoretical
framework, again, in line with the idea of a design and implementation grounded
in a solid theoretical understanding.

\subsection{Results}
Chapter~\ref{chapter_results} presents a quantitative and
qualitative performance assessment of the implemented front- and back-end
subsystems. It also presents the overall integrated MT System. 

\subsection{Conclusion}
Chapter~\ref{chapter_conclusion} ties together the project by
relating the outcomes of the various Chapters back to the initial objectives. 
It reflects on the various subsystem implementations and explores possible
improvements and avenues for future work within the different modules of the
front- and back-end subsystems.

