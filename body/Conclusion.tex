\chapter{Conclusion}
This Chapter discusses the Conclusions drawn from the Results
and Analyses of the Detection and Tracking approaches when applied to Image
Sequences, as presented in Chapter~\ref{chapter_results}.

\section{Template Matching Tracker}
The Simple Template Tracker implementation was only slightly effective in
managing to track even the simplest of sequences. It is clear that Simple Template
Matching applied to every frame of a sequence does the necessary built-in
flexibility towards even subtle changes in target appearance between frames.

The Adaptive Template Tracker implementation showed some promise in solving the
tracking problem. Where the Simple Template Tracker failed in tracking the
Taxi in the Hamburg Taxi Sequence, The adaptive Tracker succeeded for the
duration of the sequence. 
The approach of updating the template with the detected region in each
$\mathbf{f}_k$ to be used in $\mathbf{f}_{k+1}$ partly allowed the tracker to
deal with changes in object appearance in subsequent frames. However, the
approach did not have a means to effectively center the detected region, hence
over long sequence the tracker would exhibit ``drift'', eventually losing the
object to be tracked as more and more background noise was added to the model at
each update.

In addition, when faced with the Challenge of Occlusion, depending on it's
similarity threshold $\tau$ and the rate at which the occlusion covers the
target between $\mathbf{f}_k$ and $\mathbf{f}_{k+1}$, the Adaptive Template
Tracker either fails to locate the target or incorporates the occlusion into
subsequent target models.

\section{Colour Co-occurrence Histogram Detector}
Still within the sphere of Detection-Based Tracking and Template Matching, a
Feature-based Co-occurrence Histogram Detector was implemented largely as an
experiment to gauge, whether Template Matching Tracking can be extended to deal
with some of the Motion Tracking Challenges. 
Picking up from where the Adaptive Template Tracker failed, the CCH Detector was
trialed to see whether it could overcome the Challenge of Occlusion.

The approach was able to detect the model of the girl obtained from
$\mathbf{f}_0$ in a subsequent frame $\mathbf{f}_{139}$ despite the target being
significantly occluded. 
The method was however very sensitive to template selection, with small
variations in template dimensions resulting in failures, and this keeping in
mind that the target -the Girl- is a significantly distinct object from it's
background. 

Furthermore, the execution time in computing the Colour Co-occurrence Histogram
feature is a significant bottleneck in the execution time of the Detector,
limiting it's applicability. 

It has potential to be used as an initial detector. Based of a database of
learned shapes, several matches can be made within the initial frame,
$\mathbf{f}_0$ of a particular sequence, once a match is made, it can be linked
with the fast and robust Mean Shift Tracker to track the objects motions for
subsequent frames in the sequence. 

\section{Mean Shift Tracker}
In terms of performance in relation to the Motion Tracking Problem and to the
Challenges in achieving effective tracking, the Mean Shift Tracker was by far
the most successful, of the trialled Kernel-Based Tracking implementations.

Its performance was tested against a variety of sequences, exhibiting various
of the outlined Challenges to Motion Tracking.

\section{Graphical User Interface}



