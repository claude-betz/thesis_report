\chapter{Conclusion}\label{chapter_conclusion}
This Chapter aims to tie together the work of the preceding Chapters
by contextualising the progression of the project with the initial objectives 
described in Chapter~\ref{chapter_introduction}.

Conclusions from the front-end and back-end development, implementations and
results are presented, followed by insights from the overall integrated system.
Finally possible future improvements and extensions to the project are addressed.

\section{Template Matching Trackers}
The simple template tracker implementation was only slightly effective in
managing to track even the simplest of sequences. It is clear that simple template
Matching applied to every frame of a sequence does the necessary built-in
flexibility towards even subtle changes in target appearance between frames.

The adaptive template tracker implementation showed some promise in solving the
tracking problem. Where the simple template tracker failed in tracking the
taxi in the Hamburg taxi sequence, The adaptive tracker succeeded for the
duration of the sequence. 
The approach of updating the template with the detected region in each
$\mathbf{f}_k$ to be used in $\mathbf{f}_{k+1}$ partly allowed the tracker to
deal with changes in object appearance in subsequent frames. However, the
approach did not have a means to effectively center the detected region, hence
over long sequence the tracker would exhibit ``drift'', eventually losing the
object to be tracked as more and more background noise was added to the model at
each update.

In addition, when faced with the challenge of occlusion, depending on it's
similarity threshold $\tau$ and the rate at which the occlusion covers the
target between $\mathbf{f}_k$ and $\mathbf{f}_{k+1}$, the adaptive template
tracker either fails to locate the target or incorporates the occlusion into
subsequent target models.

\section{Colour Co-occurrence Histogram Detector}
Still within the sphere of detection-based tracking and template matching, a
feature-based co-occurrence histogram detector was implemented largely as an
experiment to gauge, whether template matching-based tracking can be extended to deal
with some of the motion tracking challenges. 
Picking up from where the adaptive template tracker failed, the CCH-detector was
trialled to see whether it could overcome the challenge of occlusion.

The approach was able to detect the model of the girl obtained from
$\mathbf{f}_0$ in a subsequent frame $\mathbf{f}_{139}$ despite the target being
significantly occluded. 
The method was however very sensitive to template selection, with small
variations in template dimensions resulting in failures, and this keeping in
mind that the target -the girl- is a significantly distinct object from it's
background. 

Furthermore, the execution time in computing the colour co-occurrence histogram
feature is a significant bottleneck in the execution time of the detector,
limiting it's applicability. 

It has potential to be used as an initial detector. Based of a database of
learned shapes, an initial match can be made within the initial frame,
$\mathbf{f}_0$ of a particular sequence, once a match is made, it can be linked
with the fast and robust mean shift tracker to track the objects motions for
subsequent frames in the sequence. 
The main appeal of this approach is that machine learning and deep learning
based methods take large amounts of data to be trained, a refined algorithm with
faster execution time and less sensitivity to tuning parameters could have
applicability in fast deployment systems, based off multi-view templates of a
desired object to detect.

\section{Mean Shift Tracker}
In terms of performance in relation to the motion tracking problem and to the
challenges in achieving effective tracking, the mean shift tracker was by far
the most successful of the trialled kernel-based tracking implementations.

This allowed for more comprehensive performance tests against a variety of
sequences, exhibiting various of the outlined challenges to motion tracking.

Throughout the experiments, the mean shift tracker showed good robustness to
challenges such as occlusion, changes in scale, camera ego motion, and  


\section{Graphical User Interface}
The GUI system implemented in the Qt framework was overall very responsive. The
main delays in the system execution were introduced from the waiting for
feedback from the back-end functions.

The final GUI implementation satisfied the initial user requirements defined in
Section~\ref{introduction_user_requirements}. As detailed in
Section~\ref{results_gui}, the GUI allowed for user selection of sequences of
interest. A user could navigate the sequences and apply one of the implemented
back-end detectors or trackers after selecting an object of interest in the
initial a frame, $\mathbf{f}_k$.


\section{Future Work}\label{future}
This Section outlines the possible improvements and areas
of future work for different subsystems of the MT system. Suggestions are based
on the conclusions drawn and knowledge obtained from the initial survey of the
literature in Section~\ref{chapter_literature_review}.

\subsection{Template matching tracker and colour co-occurrence histogram
detector}
The adaptive template matching trackers showed potential in tracking templates for
short sequences, it's lack of generalisation however made it incapable of
application to more complex sequences. 
The CCH-detector was an experiment at defining a deformable template
and it showed promise in detection despite occlusion, given that a major
drawback was the execution time of the searches, possible future work could be
done in speeding up of the CCH-detector using parallelisation. The computation
of the CCH descriptor is easily an embarrassingly parallel problem as the
work-load can be split across either of the descriptors dimensions, be it
quantisation bins or quantisation distances. 
There is still likely a limit to the effectiveness of this as the configuration
was relatively sensitive to changes to the $n_c$ and $n_d$ parameters which is
undesirable in a real world application.

\subsection{Mean shift tracker}
Mean shift tracking remains an active field of research. The mean shift tracker
(MST) was the most successful of the trialled kernel-based tracker implementations, and
accordingly has the most room for extension.

The (MST) implementation, has static kernel dimensions. A clear first
step to optimisation would be to add dynamic dimensioning for the kernel region
as the scale of the object of interest changes, this would mean that less
background noise is incorporated when computing candidate PDFs for calculating
the mean shift vector.

The MST is not limited to descriptors based on colour low-level features.
Section~\ref{results_speed} showed that the MST had difficulty when operating on
object and backgrounds that were similar within the RGB colour-space. A solution
to this could be to use different of a combination of different low-level
features. For example, despite an object and background being similar in colour,
they may vary in texture. Humans often spot camouflaged animals due sometimes
slight changes in texture. 
Defining the PDF of the MST based on texture features or a combination of RGB
and texture could allow sufficient discriminative power within the feature space
for the algorithm differentiate a blue fish against a blue backdrop.

When faced with challenges such as complete occlusion or target overlap with
similar objects, the MST performed poorly peaking in the error metric or failing
to keep track of the object entirely. A possible refinement could be to augment
the tracker with Bayesian algorithms. As detailed in
Section~\ref{literature_review_point}, Bayesian estimation algorithms compute
the posterior density of a parameter/state of interest. In the tracking
problem, this parameter/state could be our object position, velocity or both in
$\mathbf{f}_k$. 
Augmenting the mean shift algorithm with the addition of state estimation  would
incorporate predictability to the system based on past states. These predictions
would prove valuable in the case of complete occlusion where the mean shift
algorithm has no means of inferring the object position. It would also likely
help with the case of track overlap as the it could be setup to take into
account direction and velocity of motion to eliminate the overlapping object as
possible target in spite of similar appearances in the selected
feature space.
There are several well researched approximations to the ideal Bayesian filter
under different constraints such as Kalman Filters, Point Mass filters, Particle
Filters etc.~\cite{Challa2011}.  


\subsection{GUI}
In terms of the GUI there is a lot room for improvement. The user flow sequences
require a more thorough testing, this would be in the form of branch testing where
each possible sequence of user actions is tested to ensure it does not fail in
certain control sequences, as only the main user flow sequences were tested, as
presented in Section~\ref{results_gui}.

In terms of rendering the selected sequences to the two frame and template views
of the GUI, the frames and templates to be displayed should scale to the current
view size rather than dynamically changing the window size. This limits the
application to lower resolution videos as the window could scale beyond the
particular screen dimensions.
Correction of this would require the implementation of a coordinate transform
to convert the dimensions from the scaled frame to to the original pixel
dimensions for the user selected kernels that the back-end subsystem operates on.



