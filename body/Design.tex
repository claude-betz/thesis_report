\chapter{Design}\label{chapter_design}
*insert summary of section content and structure*

\section{System Level Design}
Development of the overall system design can conveniently be divided into
Front-End and Back-End. The Front-End system is the point of interaction for the
system user. It is GUI application that thereby satisfying the User Requirements
of the system. The Back-End consists of the algorithms enabling the tracking of
objects of interest within video, thereby addressing the Technical
Specifications of the system.

The Overall System Design is summarised by the Block Diagram in Figure
\ref{motion_tracking_subtasks}.

\Figure[width=0.8\columnwidth]{Block Diagram of System Overview\label{motion_tracking_subtasks}}{design_system_overview}

\section{User Requirements}
The end-user requirements of the proposed system are elaborated upon below:
\begin{itemize}
    \item The GUI application should have a menu that allows for a system
        user to select a video or image sequence of interest.
    \item Upon the selection of an image sequence, the system will allow for a
        user to specify an object or point of interest within the first sequence
        frame for tracking in subsequent frames.
    \item The system should reliably output a bounding box around the user
        specified point or object of interest in subsequent frames of the
        sequence. 
    \item At termination of the sequence, snapshots of the bounding box
        surrounding the relevant point of interest or object should be stored in
        a user selected directory.
    \item
    \item 
\end{itemize}

\section{Specification}
This section aims to present functional specifications, set out in the
project brief.

\subsection{Environmental Requirements}


\subsection{Non-functional Requirements}


\subsection{Feature Specifications}


\subsection{Use Cases}


This following Section are concerned with the detailed design of of the Front- and Back-end
subsystems of the Motion Tracking System.

\section{Front-End Design}
The Front-End is the GUI that user's of the Motion Tracking System interact with 
It interacts with the underlying functionality implemented in the Back-End to
satisfy the System User Requirements. 

\section{Back-End Design}
As indicated by the indicated by the literature in Section
\ref{literature_review_general_approach}. Three variants of kernel based object
trackers were implemented.

\subsection{Template Matching Tracker}

\subsubsection{Simple Template Tracking}

\subsubsection{Adaptive Template Tracking}


\subsection{Development Environment and Tech-Stack}
This section details underlying Tech-Stack of the Motion Tracking System. The
purpose of this section is to ensure easy reproducibility of the implemented
system.

\subsubsection{}

\section{Integration Design}
This section details the specifics of how 





