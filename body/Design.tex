\chapter{Design}\label{design}
*insert summary of section content and structure*

\section{System Level Design}
Development of the overall system design can conveniently be divided into
Front-End and Back-End. The Front-End being a GUI with which users interact 
of interact with the system, thereby satisfying the User Requirements of the
system.
The Back-End consists of the algorithms enabling the tracking of objects of
interest within video, thereby addressing the Technical Specifications of the
system.

The Overall System Design is summarised by the Block Diagram below: 

\Figure[width=1\columnwidth]{Block Diagram of System Overview\label{motion_tracking_subtasks}}{design_system_overview}

\section{User Requirements}
\begin{itemize}
    \item
    \item
    \item
    \item
\end{itemize}

\section{System Specifications}
This section aims to present and flesh out the specifications, set out in the
project brief.


\subsection{Environmental Requirements}

\subsection{Non-functional Requirements}

\subsection{Feature Specifications}

\subsection{Use Cases}



\section{Software Design}


\subsection{Front-End Design}
The Front-End is the GUI that user's of the Motion Tracking System interact with 
It interacts with the underlying functionality implemented in the Back-End to
satisfy the System User Requirements. 


\subsection{Back-End Design}
As indicated by the indicated by the literature in Section
\ref{literature_review_general_approach}. The Software underlying motion is broken down into three

\section{Development Environment and Tech-Stack}
This section details underlying Tech-Stack of the Motion Tracking System.



