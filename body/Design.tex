\chapter{Design}\label{chapter_design}
*insert summary of section content and structure*

\section{User Requirements}
The end-user requirements of the proposed system are elaborated upon below:
\begin{itemize}
    \item The GUI application should have a menu that allows for a system
        user to select a video or image sequence of interest.
    \item Upon the selection of an image sequence, the system will allow for a
        user to specify an object or point of interest within the first sequence
        frame for tracking in subsequent frames.
    \item The system should reliably output a bounding box around the user
        specified point or object of interest in subsequent frames of the
        sequence. 
    \item At termination of the sequence, snapshots of the bounding box
        surrounding the relevant point of interest or object should be stored in
        a user selected directory.
    \item
    \item
\end{itemize}

\section{Specification}
This section aims to present functional specifications, set out in the
project brief.

\subsection{Use Cases}
This following Section are concerned with the detailed design of of the Front- and Back-end
subsystems of the Motion Tracking System.

\section{System Level Design}
Development of the overall system design can conveniently be divided into
Front-End and Back-End. The Front-End system is the point of interaction for the
system user. It is GUI application that thereby satisfying the User Requirements
of the system. The Back-End consists of the algorithms enabling the tracking of
objects of interest within video, thereby addressing the Technical
Specifications of the system.

The Overall System Design is summarised by the Block Diagram in Figure~\ref{motion_tracking_subtasks}.
\Figure[width=0.8\columnwidth]{Block Diagram of System Design\label{fig:design_overview}}{design_system_overview}

Figure~\ref{fig:design_overview} highlights the communication between the two MOT
subsystems, in a general use case. The System is controlled via a user
friendly GUI which allows for convenient selection of parameters such as
image sequence, algorithm etc.

An in depth treatment of the Front- and Back-end follows:

\subsection{Front-End Design}
The Front-End is the GUI that user's of the Motion Tracking System interact with 
It interacts with the underlying functionality encapsulated in the Back-End API to
satisfy the System User Requirements. 

As the functionality of the system is not, the tasks that the front
end must achieve can be easily extracted from the user requirements. The GUI
application should afford a User the following functions.
\begin{itemize}
    \item A User should be able to load a desired image sequence
    \item A User should be able play, pause and iterate the image sequence 
    \item A User should be able to select one of the implemented tracking
        Algorithms and apply it.
    \item A User should be able to specify a directory in results of the
        relevant Algorithm are stored.
\end{itemize}

In line with these requirements the following, the following markups for the
GUIs were devised.

*Insert GUI Markups* 

\subsection{Back-End Design}
As indicated by the indicated by the literature in
Section~\ref{literature_review_general_approach}. 

\subsection{Integration}
The development of the Front- and Back-end will occur in isolation during the
early stages of the design. This is due to the face that the subject matter of
the two sub-modules is largely unrelated.  To ensure that this occurs smoothly,
it is necessary to have a clear understanding of exactly how the two subsystems
interact, essentially what API end-points they must avail each other.

The following UML diagram summarises the class and interface dependencies of the
MOT system.

\newpage
\Figure[width=1\columnwidth, angle=90, scale=1.2]{Block Diagram of SystemDesign\label{fig:design_uml}}{design_uml} 
\newpage
As touched upon in Section~\ref{design_back_end}, the design in
Figure~\ref{fig:design_uml} attempts to stick firmly to the Object Oriented
Programming and particularly to leverage the concepts of Inheritance and
Polymorphism to maintain a clean and easily extendible code base, without
unnecessary duplication of functionality.  Thought was given to adequately
abstract away the low level complexity especially in developing the Back-end to
allow for easy implementation. 



\subsection{Template Matching Tracker}

\subsubsection{Simple Template Tracking}

\subsubsection{Adaptive Template Tracking}


\subsection{Development Environment and Tech-Stack}
This section details underlying Tech-Stack of the Motion Tracking System. The
purpose of this section is to ensure easy reproducibility of the implemented
system.

\section{Integration Design}
This section details the specifics of how 





