\chapter{Design}\label{chapter_design}
In this Chapter we aim to document the design process of the MOT system. It does so
while in line with the User Requirements detailed in 
Section~\ref{introduction_user_requirements}. 
We then detail the Design from a higher-level System Overview in terms of a
delineation of the Front- and Back-end subsystems. The detailed the design of
these Subsystems is then presented with care given to clearly define their
interfaces and dependencies with each other, this allows for independent
development of the Subsystems in the early stages of the Development Life cycle,
and should help result in a smooth Integration Phase.

This following Section is concerned with the System Level design of the
Front- and Back-end subsystems of the Moving Object Tracking (MOT) System.

\section{System Level Design}
Development of the overall system design can conveniently be divided into
Front-End and Back-End. The Front-End system is the point of interaction for the
system user. It is GUI application that thereby satisfying the User Requirements
of the system. The Back-End consists of the algorithms enabling the tracking of
objects of interest within video, thereby addressing the Technical
Specifications of the system.

The Overall System Design is summarised by the Block Diagram in
Figure~\ref{motion_tracking_subtasks}.  

\Figure[width=0.8\columnwidth]{Block Diagram of System Design\label{fig:design_overview}}{design_system_overview}

Figure~\ref{fig:design_overview} highlights the communication between the two MOT
subsystems, in a general use case. The System is controlled via a user
friendly GUI which allows for convenient selection of parameters such as
image sequence, algorithm etc.

An in depth treatment of the Front- and Back-end follows

\subsection{Front-End Design}
The Front-End is the Graphical User Interface (GUI) that Users of the MOT
System interact. 
It interfaces with the underlying functionality encapsulated in the Back-End API to
satisfy the System User Requirements. 

The tasks that the front end must achieve relate directly to the User
Requirements. The basic GUI application should afford a User the following functions.
\begin{itemize}
    \item A User should be able to load a desired image sequence
    \item A User should be able play, pause and iterate the image sequence 
    \item A User should be able to select one of the implemented tracking
        Algorithms and apply it.
    \item A User should be able to specify a directory in results of the
        relevant Algorithm are stored.
\end{itemize}

In line with these requirements, Figure~\ref{design_gui}

\Figure[width=1\columnwidth]{Graphical User Interface Mockup\label{fig:design_gui}}{design_gui_mockup} 

\subsection{Back-End Design}
As indicated by the indicated by the literature in
Section~\ref{literature_review_general_approach}. 

\subsection{Integration}
The development of the Front- and Back-end will occur in isolation during the
early stages of the design. This is due to the face that the subject matter of
the two sub-modules is largely unrelated.  To ensure that this occurs smoothly,
it is necessary to have a clear understanding of exactly how the two subsystems
interact, essentially what API end-points they must avail each other.

The UML diagram in Figure~\ref{fig:design_uml} summarises the Class and Interface dependencies of the
MOT system.

\newpage
\Figure[width=1\columnwidth, angle=90, scale=1.2]{UML Diagram Outlining the MOT System Class Structure\label{fig:design_uml}}{design_uml} 
\newpage

As touched upon in Section~\ref{design_back_end}, the Design in
Figure~\ref{fig:design_uml} attempts to stick firmly to the Object Oriented
Programming paradigm with the intention to leverage the concepts of Inheritance and
Polymorphism. The rationale behind this is to be able to maintain a clean and
easily extendible code base, without unnecessary duplication of functionality.
Thought was given to adequately abstract away the low level complexity
especially in developing the Back-end Algorithms to allow for easy integration
towards the end of the development life-cycle. 

\subsection{Development Environment and Tech-Stack}
This section details underlying Tech-Stack of the Motion Tracking System. The
purpose of this section is to ensure easy reproducibility of the implemented
system.


