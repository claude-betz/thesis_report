\chapter{Design}\label{chapter_design}
In this Chapter we aim to document the design process of the MOT system. It does so
while in line with the User Requirements detailed in 
Section~\ref{introduction_user_requirements}. 
We then detail the Design from a higher-level System Overview in terms of a
delineation of the Front- and Back-end subsystems. The detailed the design of
these Subsystems is then presented with care given to clearly define their
interfaces and dependencies with each other, this allows for independent
development of the Subsystems in the early stages of the Development Life cycle,
and should help result in a smooth Integration Phase.

This following Section is concerned with the System Level design of the
Front- and Back-end subsystems of the Moving Object Tracking (MOT) System.

\section{System Level Design}
Development of the overall system design can conveniently be divided into
Front-End and Back-End. The Front-End system is the point of interaction for the
system user. It is GUI application that thereby satisfying the User Requirements
of the system. The Back-End consists of the algorithms enabling the tracking of
objects of interest within video, thereby addressing the Technical
Specifications of the system.

The Overall System Design is summarised by the Block Diagram in
Figure~\ref{fig:motion_tracking_subtasks}.  

\Figure[width=0.8\columnwidth]{Block Diagram of System Design}{design_system_overview}

Figure~\ref{fig:design_system_overview} highlights the communication between the two MOT
subsystems, in a general use case. The System is controlled via a user
friendly GUI which allows for convenient selection of parameters such as
image sequence, algorithm etc.

An in depth treatment of the Front- and Back-end follows. The MOT system is
developed in the Python3 programming language. 

\subsection{Back-End Design}\label{design_back_end}
The focus of the back-end design will be in the implementation of the various
approaches to Object Detection and Motion Tracking developed in
Chapter~\ref{chapter_theoretical_framework}.
Emphasis is placed on minimal reliance on libraries for functional
implementations.

The back-end structure will follow the Object Oriented Programming paradigm with the intention
to leverage the concepts of Inheritance and Polymorphism. The rationale behind
this is to be able to maintain a clean and easily extendible code base, without
unnecessary duplication of functionality. Thought was given to adequately
abstract away the low level complexity especially in developing the Back-end
Algorithms to allow for easy integration towards the end of the development
life-cycle. 

In line with this, the UML diagram in Figure~\ref{fig:design_uml} goes into
detail about the proposed Class structure of the back end and also highlights how the 
this functionality will interface with the front end to create the MOT system.

\Figure[width=1.2\columnwidth, angle=90, scale=1.2]{UML Diagram Outlining the MOT System Class Structure}{design_uml} 

The Software dependencies within the back-end consist of the following python3
libraries:
\begin{itemize}
    \item \textit{numpy}
    \item \textit{opencv}
    \item \textit{sklearn}
\end{itemize}

The simple and adaptive TM-Tracker class variants are the only back-end class
with a dependency on the \textit{opencv} library which it makes use of for it's
highly optimised implementation of the Simple Template Matching Algorithm used
within

The CH-Detector class' main dependency is the \textit{numpy} library, it makes
use of the \textit{sklearn} library for it's implementation kmeans clustering
algorithm.

The MS-Tracker class has the \textit{numpy} library as it's sole dependency.

\subsection{Front-End Design}
The Front-End is the Graphical User Interface (GUI) that Users of the MOT
System interact. 
It interfaces with the underlying functionality encapsulated in the Back-End API to
satisfy the System User Requirements. 

The tasks that the front end must achieve relate directly to the User
Requirements. The basic GUI application should afford a User the following functions.
\begin{itemize}
    \item A User should be able to load a desired image sequence
    \item A User should be able play, pause and iterate the image sequence 
    \item A User should be able to select one of the implemented tracking
        Algorithms and apply it.
    \item A User should be able to specify a directory in results of the
        relevant Algorithm are stored.
\end{itemize}

In line with these requirements, Figure~\ref{fig:design_gui_mockup} details a
simple mockup of a possible GUI for the proposed MOT System. When integrated
with the system back-end, the GUI will afford a user functionality in
line with the System User Requirements in outlined in
Section~\ref{introduction_user_requirements}.

\Figure[width=0.8\columnwidth]{Graphical User Interface Mockup}{design_gui_mockup} 

\subsection{Integration}
The development of the Front- and Back-end will occur in isolation during the
early stages of the design. This is due to the face that the subject matter of
the two sub-modules is largely unrelated. To ensure that this occurs smoothly,
it is necessary to have a clear understanding of exactly how the two subsystems
interact, essentially what API end-points they must avail each other.

Figure\ref{fig:design_uml} details that the different Detector and Tracker
implementations will be combined into a library, which is imported into the GUI
application.
The GUI simply instantiates Detector or Tracker objects and interacts with them
through the respective public methods as the UML diagram outlines.

In this way, the front-end GUI can implement the input and output (I/O) functionality
independent of the back-end. This decoupling of the I/O is desirable as in the
scenario of the back-end functions being employed by an alternative application,
it would would have the liberty of defining it's own I/O behaviour. 


\subsection{Development Environment and Tech-Stack}
This section details underlying Tech-Stack of the Motion Tracking System. The
purpose of this Section is to outline the various software tools used in
designing the MOT Systems.


