\chapter{Design}\label{chapter_design}
In this chapter we aim to document the design process of the MT system. It does so
while in line with the user requirements detailed in 
Section~\ref{introduction_user_requirements}. 
We then detail the design from a higher-level system overview in terms of a
delineation of the front-end and back-end subsystems. The detailed the design of
these subsystems is then presented with care given to clearly define their
interfaces and dependencies with each other, this allows for independent
development of the subsystems in the early stages of the development life cycle,
and should help result in a smooth integration phase.

This following Section is concerned with the system level design of the
front- and back-end subsystems of the Motion Tracking (MT) System.

\section{System Level Design}
Development of the overall system design can conveniently be divided into
front-end and back-end. The front-end system is the point of interaction for the
system user. It is GUI application that thereby satisfying the User Requirements
of the system. The back-end consists of the algorithms enabling the tracking of
objects of interest within the relevant image sequence.

The overall system design is summarised by the block diagram in
Figure~\ref{fig:design_system_overview}. It highlights the communication between the
two MT subsystems at a general level. The system is controlled via a front-end user
friendly GUI which allows for convenient selection of parameters such as
image sequence, algorithm etc. These are communicated to the back-end
algorithms that compute the result and relay it back to the front-end for display to
the system user.

\Figure[width=0.8\columnwidth]{Block diagram of system design}{design_system_overview}

Before detailed development of the front- and back-end subsystems can progress, it is necessary to
have a clear understanding of what tools are to used for either implementation to
ensure a smooth integration phase.

\section{Development Environment and Tech-Stack}
This Section details underlying tech-stack of the MT System. The
purpose of this section is to outline the various software tools to be used in
designing the MT System with the objective of facilitating smooth system integration and
portability. 

\subsection{Programming language}
The Python3.6 programming language is used for the front- and
back-end implementation.

Python virtual environments is used in order to guarantee portability of the
final product. It creates isolated installation libraries that ensure there are
no version clashes by the changes to the global python packages during the
project development life cycle.

The installations of the various packages and libraries were handled by the pip3
Python3 package manager.

\subsection{Relevant libraries}
The relevant Python3 libraries chosen for the development of the MT system are
outlined below. The Implementation Chapter presents their use in detail.
\begin{itemize}
    \item \textbf{NumPy}: The NumPy library is used for it's high performance
        multidimensional array object and the efficient operations it provides.
    \item \textbf{PyQt5}: This is the chosen library for the GUI system, it
        provides high-level APIs for GUI development.     
    \item \textbf{OpenCV}: This is a computer vision library aimed at
        real-time computer vision, it is used for it's highly optimised template
        matching function, and to read in images sequences in the front-end GUI
    \item \textbf{scikit-learn}: This is a machine learning library that is
        used for it's implementation of the kmeans clustering algorithm.
\end{itemize}

Based on these tools, an in depth treatment of the front- and back-end subsystem
design follows. 

\section{Back-End Design}\label{design_back_end}
The focus of the back-end design will be in the implementation of the various
approaches to object detection and motion tracking developed in
Chapter~\ref{chapter_theoretical_framework}.
Emphasis is placed on minimal reliance on libraries for functional
implementations.

\subsection{Object oriented approach}
The back-end structure will follow the object oriented programming paradigm with
the intention to leverage the concepts of inheritance and polymorphism. The
rationale behind this is to be able to maintain a clean and easily extendible
code base, without unnecessary duplication of functionality. Thought was given
to adequately abstract away the low level complexity especially in developing
the back-end algorithms to allow for easy integration towards the end of the
development life-cycle. 

In line with this, the UML diagram in Figure~\ref{fig:design_uml} goes into
detail about the proposed class structure of the back-end and also highlights how the 
this functionality will interface with the front end to create the MT system.

\Figure[width=1.2\columnwidth, angle=90, scale=1.2]{UML diagram outlining the MT system class structure}{design_uml} 

\subsection{Software dependencies}
This subsection describes the software dependencies within the back-end
subsystem.

The simple and adaptive template matching variants in the TM-tracker class
variants are the only back-end class with a dependency on the \textbf{OpenCV}
library which it makes use of for it's highly optimised implementation of the
template matching algorithm.

The CH-Detector class' main dependency is the \textbf{NumPy} library, it makes
use of the \textbf{scikit-learn} library for it's implementation kmeans clustering
algorithm.

The MS-Tracker class has the \textbf{NumPy} library as it's sole dependency.

\section{Front-End Design}
The front-end is the graphical user interface (GUI) that users of the MT
system interact with. 
It interfaces with the underlying functionality encapsulated in the back-end API to
satisfy the system user requirements. 

\subsection{GUI requirements}
The tasks that the front end must achieve relate directly to the user
requirements outlined in Section~\ref{introduction_user_requirements}. The basic
GUI application should afford a user the following functions.  

\begin{itemize}
    \item A user should be able to load a desired image sequence
    \item A user should be able play, pause and iterate the image sequence 
    \item A user should be able to select one of the implemented tracking
        algorithms and apply it.
    \item A user should be able to specify a directory in results of the
        relevant algorithm are stored.
\end{itemize}

Figure~\ref{fig:design_gui_mockup} details a simple mock-up of a possible GUI
for the proposed MT system. The proposed layout is simple, it proposes menus for
file and algorithm selection, two views to display tracking results to the user
and several buttons that allow a user to navigate a particular sequence
When integrated with the system back-end, the GUI functionality will directly
satisfy the defined system user requirements.

\Figure[width=0.8\columnwidth]{Graphical user interface mock-up}{design_gui_mockup} 

\subsection{GUI programming}
In line with the general model-view-controller architecture, the GUI consists
of:
\begin{itemize}
    \item \textbf{A model}: this is the application object responsible for the
        dynamic behaviour of the GUI through the event loop.
    \item \textbf{Views}: these are elements from which users are presented
        information from the application. 
    \item \textbf{Controllers}: these are elements with which users interact with
        the application. They emit signals.
\end{itemize}

\subsection{Software dependencies}
This subsection outlines the software dependencies of the front-end subsystem.
The main front-end dependency is the \textbf{PyQt5} Python3 library for the
rendering of the GUI and all it's constituent elements.

The \textbf{OpenCV} library is used by the GUI for the input of image sequences,
and output of processed image sequences to particular directories.

While the front-end is developed in isolation, the back-end subsystem can be
seen as a dependency to the GUI in the integration phase, as detector and
tracker modules are imported from it and become instance variables of the GUI
class object.

\section{Integration}
The development of the front- and back-end will occur in isolation during the
early stages of the design. This is due to the face that the subject matter of
the two sub-modules is largely unrelated. To ensure that this occurs smoothly,
it is necessary to have a clear understanding of exactly how the two subsystems
interact, essentially what API end-points they must avail each other.

Figure~\ref{fig:design_uml} details that the different detector and tracker
implementations will be combined into a library, which is imported into the GUI
application.
The GUI simply instantiates detector or tracker objects and interacts with them
through the respective public methods as the UML diagram outlines.

In this way, the front-end GUI can implement the input and output (I/O) functionality
independent of the back-end. This decoupling of the I/O is desirable as in the
scenario of the back-end functions being employed by an alternative application,
it would would have the liberty of defining it's own I/O behaviour. 


