\begin{centerpage}{Abstract}
 
Motion tracking is a very application specific endeavour. For example, an
    algorithm that reliably tracks the motion of a pedestrian in a city scene may fall
    short at detecting the motion of an animal in the woods. A military targeting
    system would have drastically different constraints to say a sports analyst
    who highlights player movement to draw attention to a particular play.
    Accordingly, there are several approaches to the problem, yet even after evaluation
    the trade-offs between various approaches, reliable tracking can still require
    considerable fine tuning of model parameters and assumptions before arrival at a
    satisfactory solution to a particular application domain. 
    Changes in appearance such as the introduction of occlusion or changes in
    illumination of a scene can all drastically affect model performance should it
    not intrinsically account for this.

The objectives of this project were to engage investigative study into the field
    of motion tracking. The outcome of this study was the implementation of a
    motion tracking (MT) system from the ground up, by way of a theory driven design
    and implementation approach. The implemented MT System consists of a
    back-end motion tracking module that is subsequently integrated into a
    front-end graphical user interface (GUI) for easy user interaction with the
    implemented functionality.
    The project also presents an application programming interface (API) with
    which developers can incorporate the implemented back end subsystem into
    their own applications.

The development of the MT System followed a well defined methodology that
    definined the various lifecycle phases and their governing process
    methodologies. 
    
The initial research presents a broad survey of the field of motion tracking,
    the outcome of which is a refined scope, and focus on kernel based tracking
    methods. Following an in depth presentation of the underlying theory
    of the template matching and mean shift kernel based tracking approaches,
    this project provides a detailed design of the delineated front and back end
    subsystems of the overall MT System. The MT System front and back end
    implementation follows organically from the specified theoretical framework
    and system design. 
    
The implementation is tested both quantitaviely and qualitatively to assess the
    front-end, back-end and overall integrated system performance. 
    Conclusions about the system and subsytem performance are then drawn based on a
    comprehensive analysis of collected test results. The efficacy in meeting the
    outlined project objectives and user requirements is gauged and possible
    improvements and future areas of work explored.




\end{centerpage}
