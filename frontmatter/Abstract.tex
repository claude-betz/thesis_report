\begin{centerpage}{Abstract}
 
Motion tracking is a very application specific endeavour. For example, An
    algorithm that reliably tracks the motion of a pedestrian in a city scene may fall
    short at detecting the motion of an animal in the woods. A military targeting
    system would have drastically different constraints to say a sports analyst
    who highlights player movement to draw attention to a particular play.
    Accordingly, there are several approaches to the problem. Even after evaluation
    the trade-offs between various approaches, this problem can still require
    considerable fine tuning of model parameters and assumptions before arrival at a
    reliable solution to a particular application domain. 
    For example, changes in appearance such as the introduction of occlusion io a scene, or changes in
    illumination of a scene can all drastically affect model performance should it
    not intrinsically account for this.

The objectives of this project were the implementation of a robust motion
    tracking system from the ground up, by way of a theory driven design and
    implementation approach. 
    This project serves as an investigative study into the field of motion tracking.
    Following a broad survey of the field in line with the project objectives,
    the project scope is narrowed down to a study of the class kernel based
    trackers.
    
The result of this study is a back-end software motion tracking module that
    is subsequently integrated into a front-end graphical user interface (GUI) for easy
    user interaction with the implemented functionality, the overall system is
    dubbed the MT System. 
    The project also presents an application programming interface (API) with which developers can incorporate the
    implemented back end subsystem into their own applications.

The development of the MT System followed a well defined methodology defining the various
    lifecycle phases and their governing process methodologies. 

Following an in depth presentation of the underlying theory of the template matching
    and mean shift kernel based tracking approaches, this project provides a
    detailed design of the delineated front and back end subsystems of the
    overall MT System. 
    The MT System front and back end implementation follows organically from the
    specified theoretical framework and system design. The results presented assess
    the front-end, back-end and overall integrated system performance both qualitatively and
    quantitatively. Conclusions about the system and subsytem performance are drawn based on a comprehensive
    analysis of collected results. The efficacy in meeting the outlined project
    objectives and user requirements is gauged and possible improvements and
    future areas of work explored.




\end{centerpage}
